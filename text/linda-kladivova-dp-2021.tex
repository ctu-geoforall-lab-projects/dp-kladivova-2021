%Balíčky
\documentclass[a4paper,10pt,twoside]{article}
\usepackage[utf8]{inputenc} %kodovani, abychom mohli jednoduse psat diakriticka pismena
\usepackage[english]{babel}
\usepackage{pdfpages}
\usepackage{pifont}
\usepackage{graphicx} %pro vkladani obrazku
\usepackage{color}
\usepackage{fancyhdr} %zahlavi a zapati
\usepackage{ifpdf}
\usepackage{amssymb}
\usepackage{booktabs}
\usepackage{subfigure}
\usepackage{titlesec}
\usepackage[multiple]{footmisc}
\usepackage{color}       % pro zvýraznění textu barvou
\usepackage{multirow} 
\usepackage{multicol}
\usepackage{amsmath}
\usepackage{sectsty}
\usepackage[justification=centering]{caption}
\usepackage[top=1.5cm, left=2cm, right=2cm, bottom=2cm, headheight=26pt, includeheadfoot]{geometry}
\usepackage[colorlinks=false,urlcolor=black]{hyperref}
\usepackage{xcolor}
\usepackage{listings}
\hypersetup{
    colorlinks=true,
    linkcolor=black,
    filecolor=black,      
    urlcolor=black,
    citecolor=black
}
\allsectionsfont{\rmfamily} 

\sectionfont{\huge}
\subsectionfont{\LARGE}
\subsubsectionfont{\Large}

\def\nazevprace{\Large{Creation of a new GRASS GIS startup mechanism}}
\def\nazevpraceEN{\large{Tvorba nového startovacího mechanismu v prostředí GRASS GIS}}

\begin{document}
\sloppy
\setlength{\parskip}{8pt}

%%  ÚVODNÍ STRÁNKA %%%%%%%%%%%%%%%%%%%%%%%%%%%%%%%%%%%%%%%%%%%%%%

\pagestyle{empty} % vypne číslování stránek na úvodní straně

\begin{center}

\LARGE
\textsc{Czech Technical University in Prague} \\
\textsc{Faculty of civil engineering} \\

\bigskip

\large
\textsc{Department of Geomatics} \\

\vspace{6ex}

\begin{figure}[hbt!] %vlozeni loga
\begin{center}
\includegraphics[width=5.5cm]{../pictures/logo_cvut.png} 
\end{center}
\end{figure}

\vspace{20ex}

\LARGE{MASTER'S THESIS}\\
\bigskip
\bigskip
\textsc{\nazevprace} \\
\smallskip
\textsc{\nazevpraceEN} \\

\mbox{}
\vfill

\normalsize
\textsc{\author} \\
\bigskip
\normalsize
\textrm{Supervisor: Ing. Martin Landa, Ph.D.} \\

\vspace{10ex}
\large
\textrm{2020 Prague} \hfill
\textrm{Bc. Linda KLADIVOVÁ} \\

\end{center}

%% 2. STRÁNKA ZŮSTANE PRÁZDNÁ

\newpage ~ \newpage
\thispagestyle{empty}

%% 3. STRÁNKA NA ZADÁNÍ
\begin{figure}
 \centering 
 \includepdf[pages=-]{../assignment/zadanidp.pdf}
\end{figure}

%% 4. STRÁNKA ZŮSTANE PRÁZDNÁ

\newpage ~ \newpage
\newpage ~ \newpage
\thispagestyle{empty}

%% 5.  STRÁNKA = ČESKÁ A ANGLICKÁ ANOTACE %%%%%%%%%%%%%%%%%%%%%%%%%%%%%%%%%

\renewcommand{\baselinestretch}{1.2} %zvetseni mezery mezi radky


\begin{Large}
\noindent ANNOTATION
\end{Large}

\large
\noindent
The existing GRASS GIS software startup mechanism could discourage new
users from further working with this software or at least make it
uncomfortable. This diploma thesis is built on the programming part
performed in the summer of 2020 within the international Google Summer
of Code program (GSoC) and uses two questionnaires to evaluate the
benefits of significant changes that have taken place. The first part
of the work focuses on a survey among intermediate users and compares
the startup mechanism of the original GRASS GIS 7.8 version with the
new solution introduced after GSoC which during normal startup cancels
the concept of the startup window and its role is taken over by Data
Catalog. The second part is oriented on newcomers and implements the
first-time mode. A survey based on a simple task further examines
whether the initial contact of the user with the software when using
the first-time mode is more pleasant or not.

\vspace{2ex}
\begin{Large}
\noindent KEYWORDS
\end{Large}

\large
\noindent
\textrm{GRASS GIS, GUI, wxPython, startup, GSoC, first-time user, software development, survey}

\mbox{}
\vfill

\begin{Large}
\noindent ANOTACE
\end{Large} 

\large
\noindent
Dosavadní startovací mechanismus softwaru GRASS GIS mohl odradit nové
uživatele od další práce s tímto softwarem nebo ji alespoň
znepříjemnit. Cílem této práce je navázat na programovací část
vytvořenou v létě 2020 v rámci mezinárodního programu Google Summer of
Code (GSoC) a pomocí dvou průzkumů vyhodnotit přínos výrazných změn,
ke kterým došlo. První část práce se zaměřuje na průzkum mezi středně
pokročilými uživateli a porovnává startovací mechanismus původní verze
GRASS GIS 7.8 s novým řešením představeným po GSoC, které ruší při
běžném startování koncept startovacího okna a tuto roli přebírá Data
Catalog. Druhá část se orientuje na nové uživatele a implementuje
tzv. "first-time" mód. Průzkumem založeným na jednoduchém úkolu dále
zkoumá, zda je počáteční kontakt uživatele se softwarem při využití
"first-time" módu příjemnější či nikoliv.

\vspace{2ex}
\begin{Large}
\noindent KLÍČOVÁ SLOVA
\end{Large}

\large
\noindent
\textrm{GUI, GRASS GIS, wxPython, startup, GSoC, first-time uživatel, vývoj softwaru, průzkum}


%% 6. STRÁNKA ZŮSTANE PRÁZDNÁ

\newpage ~ \newpage
\thispagestyle{empty}

%% 7. STRÁNKA = DECLARATION OF AUTHORSHIP %%%%%%%%%%%%%%%%%%%%%%%%%%%%%%%%%

\newpage
\mbox{}
\vfill
\begin{Large}
\noindent DECLARATION OF AUTHORSHIP
\end{Large}

I hereby declare that the work presented here is, to the best of my
knowledge and belief, the original result of my own investigations,
except as acknowledged. All direct or indirect sources used are
acknowledged as references.  \vspace{3ex}

\noindent In Prague ................................... \hfill ................................................

%% 8. STRÁNKA ZŮSTANE PRÁZDNÁ

\newpage ~ \newpage
\thispagestyle{empty}


%% 9. STRÁNKA = ACKNOWLEDGEMENT %%%%%%%%%%%%%%%%%%%%%%%%%%%%%%%%%

\newpage
\mbox{}
\vfill
\begin{Large}
\noindent ACKNOWLEDGEMENT
\end{Large}

Firstly I would like to thank my parents very much for their support
during my studies. Then I would like to express my great thanks to
Martin Landa who inspired me and still inspires me a lot on my way of
becoming a professional python developer. He was also at the beginning
of my participation in the Google Summer of Code program. In the GRASS
GIS open-source environment, I met an amazing community of incredibly
inspiring people. Among those people, I would like to thank especially
Anna and Vaclav Petras who were a great support to me during GSoC and
even later on and brought a lot of valuable advice to this
work. Finally, I would like to thank all GRASS users, whether complete
beginners or advanced, who have participated in the surveys created in
this work. They greatly contributed to the decision on how to increase
the user-friendliness of GRASS GIS.

%% 10. STRÁNKA ZŮSTANE PRÁZDNÁ

\newpage ~ \newpage
\thispagestyle{empty}


%% 11. a 12. STRÁNKA = OBSAH A SEZNAM OBRAZKU %%%%%%%%%%%%%%%%%%%%%%%%%%%%%%%%%%
\newpage

\tableofcontents %obsah
\newpage
\listoffigures %seznam obrazku

\thispagestyle{empty}
\newcommand{\obrazek}[1]{(viz obr. \ref{#1})} %specialni reference na obrazek

\newpage
\pagestyle{fancy}

%% NASTAVENI VZHLEDU STRANEK (ZAHLAVI A ZAPATI)

% zajistí, že se názvy kapitol a sekcí nebudou sázet velkými písmeny
\renewcommand{\sectionmark}[1]{\markright{\ #1}}

\fancyhf{} % smaže aktuální nastavení záhlaví a zápatí
\renewcommand{\headrulewidth}{0.4pt} % vrchní linka
\renewcommand{\footrulewidth}{0.4pt}  %  spodní linka
\addtolength{\voffset}{-0.4cm}

 %záhlaví
\fancyhead[LE, LO]{{\includegraphics[width=1cm]{../pictures/logo_cvut.png} }
   {\textsc{\small {CTU in Prague}} }} %logo skoly
\fancyhead[RE, RO]{\nouppercase{\rightmark}}
   
 %zápatí
\fancyfoot[RO, LE]{{\textsc{\small \thepage}}}

\fancypagestyle{plain}{
  \fancyhead{} % na prázdných stránkách nechci záhlaví
  \renewcommand{\headrulewidth}{0pt} % ani linku
}


%% -------<<< Chapter: Introduction >>>-------\\%%%%%%%%%%%%%%%%%%%%%%%%%%%%%%%%%%%%
\newpage
\vspace*{-1cm}
\pagestyle{fancy}
\fancyhead[RE, RO]{\fancyplain{}{\small \sl{Introduction}}}
\section{Introduction}
\large
\setcounter{page}{17}  % nastaví čítač stránek od stránky Úvod na stránku č. 13

\noindent According to the evaluation of the GISGeography journal
\cite{gisgeography}, GRASS GIS (Geographic Resources Analysis Support
System) is one of the best software we can meet in the world of
open-source software focused on Geographic Information System (GIS) in
terms of numerical analysis. The history of GRASS, built for vector
and raster geospatial management, geoprocessing, spatial modeling, and
visualization, dates back to 1982 when its development was started by
the United States military. At the time of writing this thesis, the
current stable version of the software is version 7.8, but as the
number suggests, the planned version 8.0, which will be released in
the spring of 2021, will introduce major changes. They are going to be
largely related to the GRASS graphical user interface (GUI).

Although, in the startup screen of version 7.8, there is a certain
effort to provide the first-time user with the maximum possible help,
the development community often encountered misunderstanding from the
ranks of users. These complaints led to the creation of an
% ML: Prague Roadmap could be in quotes, but for sure - reference is missing 
implementation proposal known as the Prague Roadmap. This proposal was
the basis for the author's participation in the global program Google
% ML: "students into" (skip student developers - what does it mean?)
Summer of Code program (GSoC) focused on bringing more student
developers into open source software development. The task of GSoC was
to change the sort of unfortunate GRASS startup mechanism so that it
would be easier for first-time users to become familiar with
GRASS. After the changes within GSoC, the startup screen -- the
% ML: find better word that "trick"
biggest trick for new users, was partially removed. The main role for
the data organization was taken over by the Data Catalog -- a tree
object, whose functionality was significantly expanded. Now, the
possibilities of managing data hierarchy components are even beyond
the options previously available in the startup screen.

This work, therefore, follows up on the complex topic of creating a
better GRASS GIS startup mechanism. The main part of the work consists
of two surveys where the first one consists of two parts. The aim of
the first part is to evaluate the benefits of significant changes
% ML: replace "final" with more suitable word (final is never final;-)
among the GRASS community and at the same time to propose a final
% ML: for "existing" user - what does it mean?
solution of the GRASS startup mechanism for the existing user so that
the old startup screen is permanently removed. The second part of the
first survey focuses on the improvement of the startup mechanism for
first-time users. Already at GSoC, it was decided that some form of
assistance to new users would need to be implemented. The second part
of the survey finds out what options of first-time help would be
preferred. The second survey conducted one month later introduces a
new special mode for first-time users (``first-time mode''), which
extends the concept of the default location with an Info Bar helping
users to manage the first steps. It examines whether users like the
newly designed mockups of the Info Bar and gives them space to share
ideas. Based on the analysis of the second survey, the Info Bar is
improved and subsequently implemented.

\newpage
\vspace*{-1cm}
\subsection{GRASS GIS}
\label{subsection:grassgis}
\noindent GRASS GIS is a cross-platform desktop geographic information
system (GIS) designed to work with geographic 2D/3D raster and vector
data with SQL-based attribute management, and vector network analysis,
both using the command line and graphical user interface
(GUI). Besides, it offers many spatial modelling algorithms, 3D
visualization, as well as image processing routines pertaining to
LiDAR and multi-band imagery \cite{NETELER2012124}. It is open-source
software published under the GNU GPL general license and managed and
developed under the Open Source Geospatial Foundation
(OSGeo). Important users of the GRASS system include, for example,
NASA, NOAA, USDA, USGS, and many environmental consulting companies
\cite{grassgis}.

The power of software stems mainly from its Unix philosophy, where the
software itself consists of a collection of more than 500 applications
called modules. Each of these modules has only one task to perform,
and the real power of the software comes when the various of these
modules begin to chain together, allowing the user to create even very
complex applications. Most of these modules are written in C, but
above the whole system, PyGRASS as an object-oriented Python
Application Programming Interface (API) stands, which hides the
complexity of GRASS and provides access to the capability of the C-API
of GRASS for geo-scientists that are not familiar with C
\cite{pygrass}.

% ML: GRASS is using versioning system since 1999 (CVS, than
% Subversion). Starting with such sentence can be misleading.
The software has been developing since January 2020 on GitHub, a web
version control using Git. Nowadays, most main changes take place in
the Python language, which also applies to the GUI, which uses the
% ML: extension -> library
wxPython extension. Besides improving the GRASS GIS startup mechanism,
in summer 2020 the community presented a new website (see Figure
\ref{fig:grass_gis}) on the occasion of its 37th birthday which offers
a curated list of tutorials in different languages and links to
videos.

\vspace{0.7cm}
\begin{figure}[hbt!]
\begin{center}
\includegraphics[width=16cm]{../pictures/grass_gis.png} 
\caption[New GRASS website's layout]{New GRASS website's layout (Source: \cite{grass})}
\label{fig:grass_gis}
\end{center}
\end{figure}

\newpage
\vspace*{-1cm}
\subsubsection{Data hierarchy in GRASS GIS}
\label{subsection:hierarchy}
\noindent
\large

\noindent While in other GIS software it is usually customary to store
work in so-called \textit{projects} containing \textit{map layers},
GRASS GIS keeps its unique data hierarchy, which has proven very
useful over the years, especially for experienced users. Every GRASS
GIS user has undoubtedly come across the following terms: Database,
Location, Mapset, and Maps. These four representations form a tree of
rules, which we can see in Figure \ref{fig:grass_data_hierarchy}.

\vspace{0.3cm}
\begin{figure}[hbt!]
\begin{center}
\includegraphics[width=14cm]{../pictures/grass_data_hiearchy.png} 
\caption[GRASS GIS 7 location structure]{GRASS GIS 7 location structure (Source: \cite{hierarchy})}
\label{fig:grass_data_hierarchy}
\end{center}
\end{figure}

\noindent The Database, which is built hierarchically at the top, has
the character of a base directory, whose usual name is
``grassdata''. It contains Locations - a collections of data with
common coordinate reference system (CRS). Therefore, Maps in the
Mapsets contained in a particular Location will always have the same
coordinate system.

It is also important to mention that when creating a location, a
mapset named PERMANENT is automatically created. Briefly speaking, the
PERMANENT mapset is used to store general spatial data which are also
accessible but write-protected to other users who are working in the
same Location as the Database owner. The PERMANENT mapset also holds
% ML: not only extent, but also resolution ("computational region" in GRASS terminology)
the default region boundary coordinate values which are very important
for raster analysis \cite{hierarchy}.

In the following text, the elements of the GRASS data hierarchy are
already taken as standard terms and the initial letters are lowercase.

\newpage
\vspace*{-1cm}
\subsection{State of Art before GSoC}
\label{sec:beforeGSoC}

\noindent Since the GRASS GIS version 7.8 (before GSoC) has
considerably complicated setup, we need to define what the term
\textit{startup mechanism} means in connection with GRASS. It
basically includes three things:

\begin {itemize}

\item the way GRASS GIS can be started. In the case of an Unix
  operating system, it is run from the command line.

\item GUI (graphical user interface) components that the user
  encounters during startup. In the case of GRASS GIS version 7.8,
  these are the splash screen, startup screen and Location
  Wizard. There may be situations where the requested mapset is locked
  (this happens if it is used by another process, or if the last
  session in this mapset ended in an error). If the running mapset is
  locked, we are first notified that there is a lock file with a
  .gislock extension in the mapset, and then asked if we want to
  delete this file. In this special situation, when running GRASS, we
  even encounter five different GUI components.  In the case of the
  version after GSoC, the mentioned components are bypassed in most
  situations and we can move straight to the third point.

\item the state instantly after startup - here we can talk e.g. about
  the state of individual Layer Manager tabs, and Map Display.

\end{itemize}

\noindent As we could notice, there are several GUI components that
need to be clarified at the outset. Let's first look at the role of
the individual GUI components in version 7.8 (before GSoC). Due to
significant changes, the author of this work made during the GSoC, the
role of the startup screen and Data Catalog in the version after GSoC
is very different compared to version 7.8. The Data Catalog takes over
the role of the startup screen. The state after GSoC is clearly
described in the section \ref{sec:afterGSoC}, so that we can evaluate
the benefits of the changes in the first part of the first survey. In
both subsections \ref{sec:beforeGSoC} and \ref{sec:afterGSoC}, the
GRASS GUI software components are arranged chronologically according
to the order in which the user encounters them.

\bigskip
\noindent \textbf {Startup screen versus splash screen}

\noindent As Ed Foster stated in 1996 \cite{foster}: ``Splash screens,
as they are commonly called, are the graphic logos that display while
the program is loading and identify the program while reminding you
about the software publisher's copyright restrictions." So, it appears
before the main software window starts and remains visible for a few
seconds. If we try to find articles on the startup screen, we will not
be very successful. Nowadays, this topic lives mainly on programming
websites such as Stack Overflow. In some software, startup screen can
mean at the same time a splash screen if no other startup screen
appears (among GIS software e.g. gvSIG, SAGA GIS). However, generally
speaking, a startup screen usually requires some initial action from
the user to set up the software.

The startup screen is the first component of the GRASS GIS version 7.8
% ML: "meet" doesn't sound good, try better wording
that we meet. It allows us to set all the above-mentioned components
except maps and, in the case of locations and mapsets, also manage
them in terms of renaming and deleting. The various historical
versions of the startup screen can be seen in Figure
\ref{fig:verze_startup}. The one on the right corresponds to version
7.5 (as well as 7.8).

%% ML: source?
\vspace{0.3cm}
\begin{figure}[hbt!]
\begin{center}
\includegraphics[width=17cm]{../pictures/verze_startup.png} 
\caption[Historical versions of the GRASS startup screen]{Historical versions of the GRASS startup screen}
\label{fig:verze_startup}
\end{center}
\end{figure}

\bigskip
\noindent \textbf {Location Wizard}

\noindent Location Wizard is a software component, which has the
character of a guide that appears when creating a new
location. Therefore, the main task of the wizard is to define the
Coordinate Reference System (CRS). It consists of four consecutive
dialog boxes. A comparison of the Location Wizard before and after
GSoC is included in subsection \ref{sec:afterGSoC}.

\bigskip
\noindent \textbf {Layer Manager and Map Display}

\noindent The peculiarity of GRASS GIS is that it does not consist of
one software window, as is usually the custom, but directly of
% ML: in section header you mentioned 'Map Display', keep terminology consistent (Display vs Window)
two. The topic of connecting Layer Manager with Map Window is, by the
way, one of the proposed topics on GSoC
\footnote{\url{https://trac.osgeo.org/grass/wiki/GSoC/2020\#GRASSGUI:Singlewindowlayout}}.

\vspace{0.3cm}
\begin{figure}[hbt!]
\begin{center}
\includegraphics[width=10cm]{../pictures/loc_wizard_sour_pred.png} 
\caption[Choosing EPSG code in Location Wizard in version 7.8]{Choosing EPSG code in Location Wizard in version 7.8 (Source: Personal collection)}
\label{fig:loc_wizard_sour_pred}
\end{center}
\end{figure}

\newpage
\noindent As explained in more detail in the GRASS documentation, the
Layer Manager provides a GUI for creating and managing maps. In Figure
\ref{fig:empty_layers1} we can notice five tabs - Layers, Console,
Modules, Data, and Python.  In the Layer Manager version 7.8, after
running GRASS we first see the Layers tab, which allows layers in Map
Display to be switched on and off. After starting the session this tab
is empty. The path to the current mapset and location can be noticed
in the top bar in the Map Display window. In this case, the current
% ML: demomapset in italics(?)
mapset is called demomapset and is located in the location named
\textit{fire\_grassdata}. GRASS GIS allows adding more Map Display
% ML: at beggining of this paragraph you are using "maps" term, now
% "layers". This need to be explained, kept consistent.
windows. We can then manage layers in these windows via the Layers
tab.

\vspace{0.3cm}
\begin{figure}[hbt!] 
\begin{center}
  \includegraphics[width=17cm]{../pictures/empty_layers1.png}
  % ML: Window vs Display
\caption[Layer Manager and Map Window in version 7.8]{Layer Manager and Map Window in version 7.8 (Source: Personal collection)}
\label{fig:empty_layers1}
\end{center}
\end{figure}

\noindent In this work we will deal only with the Data tab, the
explanation of other tabs can be found in the documentation
\footnote{\url{https://grass.osgeo.org/grass79/manuals/wxGUI.html}}.
 
\bigskip
\noindent \textbf {Data Catalog}

\noindent When we want to display the data located in the
\textit{current mapset} (equal to active mapset), we need to go to the
Data tab. In this tab provided in Figure \ref{fig:data_catalog_pred},
there is a Data Catalog in the middle, and the toolbar in the upper
part. Within the Data Catalog we are allowed to work with maps through
the context menu - to display, rename and delete them, display their
metadata, or to copy them to another mapset. However, the Data Catalog
in version 7.8 does not provide management of mapsets, locations, or a
database. This management is sort of hidden in the Settings/GRASS
working environment tab.

The data in the current mapset are firmly connected to the Map Display
window. If we want to see maps from a different mapset we can find in
the mapset context menu the option for switching. We are able to
switch between mapsets in the same location or between mapsets in
different locations. If we move the data to the mapset in another
location with a different CRS, the projection takes place. GRASS GIS
does not use ``on the fly'' transformation.

\vspace{0.3cm}
\begin{figure}[hbt!] 
\begin{center}
\includegraphics[width=17cm]{../pictures/data_catalog_pred.png} 
\caption[Data Catalog in version 7.8]{Data Catalog in version 7.8 (Source: Personal collection)}
\label{fig:data_catalog_pred}
\end{center}
\end{figure}

\newpage
\vspace*{-1cm}
\subsection{Prague Roadmap}
\label{section:Prague Roadmap}
\noindent
\large The data hierarchy described in subsubsection
\ref{subsection:hierarchy} proves its worth especially when used by
experienced users of GRASS, however, for complete beginners, it is
rather confusing. Especially if the above-mentioned main components
(database, location, and mapset) have to be defined right at the start
of the software in the startup screen, which has been standard since
version 6.4.  Although, in the startup screen of version 7.8, there is
a certain effort to provide the first-time user with the maximum
possible help (short description of data hierarchy, Help button), the
development community often encountered misunderstanding from the
ranks of users
\footnote{\url{https://trac.osgeo.org/grass/ticket/3474}}. The
disadvantages of GRASS mentioned in the evaluation of the GISGeography
journal \cite{gisgeography} are also largely related to the current
startup mechanism. According to this source, the main disadvantage of
GRASS version 7.8 is mainly the clunky and dated user interface,
defining projects on start-up and steep learning curve to get started.

The general question, on which answer the community has still not
fully agreed over the years, is whether to keep existing data
hierarchy at all or change the whole concept and use only
\textit{project} and \textit{map} terms, which is the usual standard
for other GIS software. The \texttt {database/location/mapset}
mechanism may indeed seem complicated at first glance, but we must
keep in mind that many later problems will be avoided by clearly
defining the CRS at the beginning and allowing only one coordinate
system within one location. From the author’s point of view, the
advantage (perhaps from the point of view of first-time users maybe
the disadvantage) is that GRASS GIS does not support ``on the fly''
transformation which shows differently projected data in the right
place on the map. It guarantees that we cannot analyze data having a
different coordinate system together, as could happen in ArcGIS or
QGIS, for example. In GRASS GIS, we also cannot get into a situation
that ``on the fly'' transformation does not occur at all, but despite
this, it is allowed to display two layers of different coordinate
systems on top of each other. This, of course, results in the
incorrect rendering of the layers in the map window (may happen
e.g. in SAGA GIS).

In 2017, the first suggestions on how to simplify this startup screen
began to appear. In terms of implementation complexity, the simplest
and most effective proposal seemed to be the proposal A3
\footnote{\url{https://trac.osgeo.org/grass/wiki/wxGUIDevelopment/New\_Startup\#ProposalA3Prague2019:Datatreeandbigbuttons}}. Based
on this proposal, the implementation proposal known as the Prague
Roadmap\footnote{\url{https://trac.osgeo.org/grass/wiki/wxGUIDevelopment/New\_Startup\#PragueRoadmap}}
was created in 2019 in Prague.  The most serious changes in Prague
Roadmap concern the Data Catalog. They lie in supporting multiple
databases, adding buttons to create existing or new databases, or
adding new actions from the context menu to a database, location, and
mapset node. Considering the disadvantages mentioned in the paragraph
above, the data hierarchy \texttt{database/location/mapset} is
preserved. In addition, this proposal introduces a new concept called
\textit{workspace} which is not directly part of a mapset, but is
associated with it.

The proposed changes in the Location Wizard are mainly related to the
clarification of the first page, better naming of the given
attributes, and speeding up the selection of the coordinate system in
the dialog. Furthermore, the proposal assumes the possibility of
filtering in the Data Catalog based on the recently selected
items. General startup GUI should be able to collect recent maps and
workspaces as well as recently used databases and workspaces. The
display of map layers in the Data Catalog is switched off by default,
the display of workspaces under the relevant mapset is switched
on. When GRASS GIS launches, the "grassdata" working directory for
storing locations, mapsets and maps should be automatically created in
a reasonable place.

Furthermore, the same Data Catalog implemented in the Data tab is
planned to be used within a startup screen. The startup screen
proposed in Prague Roadmap consists of only one startup page, which
has a Data Datalog in the center, and a toolbar with big buttons for
creating or defining new or existing data components, such as a
location or mapset.


\subsection{State of Art after GSoC}
\label{sec:afterGSoC}

\noindent The changes that occurred during the Google Summer of Code
are fundamental. At first, developers did not plan to remove the
startup screen. The changes were to be based on Proposal A3, which
planned to improve the Data Catalog, but also suggested that the same
Data Catalog, which is available in the Data tab after starting the
session, will be part of the startup screen. During the
implementation, however, all the functionality of the startup screen,
including switching mapsets as well as locations and databases, was
moved to the Data Catalog, so it no longer made sense to keep this
notion. At least definitely not the form in which it is in version
7.8.

\bigskip
\noindent \textbf {Data Catalog}

\noindent The main goal of GSoC was to improve the Data Catalog in
such a way that enables a user-friendly organization of work. This is
mainly about creating, renaming, and deleting mapsets and locations,
which was previously possible through the startup screen. However,
completely new possibilities have been added. GRASS GIS allows a user
to add multiple databases. New management icons offer intuitive
creation of the mentioned data components. In the context menu, there
are also new options for deleting several mapsets or
locations. Though, probably the most visible changes are those related
to the graphical representation of the Data Catalog. Next to the
individual components, we can notice small icons distinguishing types
of data hierarchy components. The Data Catalog also informs about
access to individual mapsets (current, in use, and a different
owner). In Figure \ref{fig:function} we can see the all
functionalities that have been newly implemented in the Data Catalog.

\vspace{0.3cm}
\begin{figure}[hbt!] 
\begin{center}
\includegraphics[width=15cm]{../pictures/funkce.png} 
\caption[New functionalities in Data Catalog]{New functionalities in Data Catalog (Source: Personal collection)}
\label{fig:function}
\end{center}
\end{figure}

\newpage
\noindent Another important thing is related to access to individual
mapsets. In version 7.8, the startup screen informs about the lock and
asks if the user wants to remove the lock. In order to completely take
over the startup screen functionality by the Data Catalog, it is also
necessary to warn the user that the mapset to which they are going to
switch is locked. How this case was solved is clear from Figure
\ref{fig:data_catalog_switch_new}. Because mapset locking was often
confused with prohibiting editing outside the current mapset, the term
was changed to \textit{mapset in use}. This term can also be seen in
the Mapset Access Info in the Data Catalog.

Another substantive step forward was the graphic change of the icon
enabling or disabling changes outside the current mapset. This icon
has newly the bitmap of a pencil (instead of bitmap of a lock in
version 7.8) and protects a user from unwanted changes. If we do not
have editing enabled, we cannot rename or delete any mapsets,
locations, and databases. We can only work with layers inside the
current mapset. However, even if we enable editing, we can never
delete a PERMANENT mapset. Similarly, we cannot delete boldly marked
\textbf{current} components in the Data Catalog.

\vspace{0.3cm}
\begin{figure}[hbt!] 
\begin{center}
\includegraphics[width=13cm]{../pictures/data_catalog_switch.png} 
\caption[Switching to mapset in use in Data Catalog]{Switching to mapset in use in Data Catalog (Source: Personal collection)}
\label{fig:data_catalog_switch_new}
\end{center}
\end{figure}


\bigskip
\noindent \textbf {Location Wizard}

\noindent This guide to creating new locations has been slightly
modified and streamlined. In particular, we can see it on the first
page ``Define a new GRASS location'' in Figure
\ref{fig:loc_wiz_1}. Checkboxes and simplified names are removed
here. Furthermore, the GRASS database can be modified.

\vspace{0.3cm}
\begin{figure}[hbt!] 
\begin{center}
\includegraphics[width=17cm]{../pictures/loc_wiz_1.png} 
\caption[Location Wizard first page before and after GSoC]{Location Wizard first page before and after GSoC (Source: Personal collection)}
\label{fig:loc_wiz_1}
\end{center}
\end{figure}

\noindent The second page is renamed to ``Select Coordinate Reference
System (CRS)''. As we can see in Figure \ref{fig:loc_wiz_2}, the
division into simple and advanced methods is abolished in the new
version. Besides, CRS can be newly specified using a WKT string.

\vspace{0.3cm}
\begin{figure}[hbt!] 
\begin{center}
\includegraphics[width=17cm]{../pictures/loc_wiz_2.png} 
\caption[Location Wizard second page before and after GSoC]{Location Wizard second page before and after GSoC (Source: Personal collection)}
\label{fig:loc_wiz_2}
\end{center}
\end{figure}

\noindent However, the essential change is related to the ``Choose
EPSG code'' page. Now it supports dynamic EPSG search and the
hyperlink which is changed dynamically according to a filter set by a
user (see Figure \ref{fig:loc_wiz_3}). The last page of the Location
Wizard called ``Summary'' remains unchanged.

\vspace{0.3cm}
\begin{figure}[hbt!] 
\begin{center}
\includegraphics[width=17cm]{../pictures/loc_wiz_3.png} 
\caption[Location Wizard third page before and after GSoC)]{Location Wizard third page before and after GSoC (Source: Personal collection)}
\label{fig:loc_wiz_3}
\end{center}
\end{figure}

\newpage
\vspace*{-1cm}
\bigskip
\noindent \textbf {Startup screen}

\noindent We can encounter three different situations when starting
the GRASS GIS ``after GSoC'' development version:

\begin{enumerate}

\item \textbf{GRASS bypasses the startup screen and starts in the
    pre-prepared default location (also ``demolocation'' in
    development slang)} (see Figure
  \ref{fig:demolocation_startup}). This situation occurs after the
  software is installed, that is when no used databases are stored in
  the settings. The default location called
  \textit{world\_latlong\_wgs84} in WGS84 coordinate system
  (EPSG:4326) shows the correct organization of the data. Base data
  are stored in a PERMANENT mapset whereas already analyzed data
  belongs to another mapset, for example to the one named after a
  user. The current implementation does not show the World map
  included in \textit{country\_boundaries} layer immediately at
  startup, it is necessary to display the map using Data Catalog (see
  Figure \ref{fig:demolocation}). The splash screen does not appear.
\item\textbf{ GRASS bypasses the startup screen if possible to start in the last used mapset} (see Figure \ref{fig:last_mapset_startup}). This situation is probably the most common. The software remembers the databases that were open when the last session was closed and opens them.  The splash screen does not appear.
\item \textbf{GRASS launches in startup screen if a mapset is not in a usable state} (was deleted or is used by another process). In this special situation, GRASS starts in the same way as in version 7.8.

\vspace{0.3cm}
\begin{figure}[hbt!] 
\begin{center}
\includegraphics[width=17cm]{../pictures/demolocation_startup.png} 
\caption[GRASS starts in the prepared default location]{GRASS starts in the prepared default location (Source: Personal collection)}
\label{fig:demolocation_startup}
\end{center}
\end{figure}

\newpage
\vspace{0.3cm}
\begin{figure}[hbt!] 
\begin{center}
\includegraphics[width=16.5cm]{../pictures/demolocation.png} 
\caption[World map as a part of the default location]{World map as a part of the default location (Source: Personal collection)}
\label{fig:demolocation}
\end{center}
\end{figure}

\vspace{0.3cm}
\begin{figure}[hbt!] 
\begin{center}
\includegraphics[width=16.5cm]{../pictures/last_mapset_startup.png} 
\caption[GRASS starts in the last used mapset]{GRASS starts in the last used mapset (Source: Personal collection)}
\label{fig:last_mapset_startup}
\end{center}
\end{figure}

\end{enumerate}

\bigskip
\noindent \textbf {Various options for starting GRASS GIS}

\large \noindent In addition to the software components that the user
encounters during startup and to the state that the software finds
itself immediately after startup, the startup mechanism also includes
the way we run the software. Because users of this software usually
work under a Unix operating system, GRASS GIS is usually run from the
command line. Advanced users often do not even run the graphical
environment and perform all geographic analyzes using the command
line. This is also why the command line window runs in the background
throughout the work with this software. In our work, however, we
mainly focus on first-time users who do not have to be
experienced. This is also basically the main reason why we try to
improve the GRASS GUI.

 \noindent To summarize, the software could be run from the command line in four ways \cite{startup}:

\noindent \texttt{:$\sim$\$ grass79} \\
\noindent Start GRASS using the default user interface. In version
7.8, the user is prompted by the startup screen. After GSoC startup
screen appears only when the mapset is not in the usable state.

\noindent \texttt{:$\sim$\$ grass79 --gui}\\
\noindent Start GRASS using the graphical user interface.  In version
7.8, the user is prompted by the startup screen. After GSoC startup
screen appears only when the mapset is not in the usable state.

\noindent \texttt{:$\sim$\$ grass79 --text} \\
\noindent Start GRASS using the text-based user interface. Appropriate
location and mapset must be either set by environmental variables or
taken from the last GRASS session.

\noindent \texttt{:$\sim$\$ grass79 --gtext} \\
\noindent Start GRASS using the text-based user interface.  In version
7.8, this option as the only one does not display the startup
screen. However, it means that the desired location and mapset must
already be set as environmental variables either by running a specific
mapset using the \texttt{:$\sim$\$ grass78
  \$HOME/grassdata/location/mapset} command, or taken from the last
GRASS session, specifically from a file in the path
\textit{\$HOME/.grass7/rc}. In version 7.9, --gtext uses startup
screen but then does not run the GUI.

The possibilities of starting GRASS GIS can be further extended by the
concept of \textit{workspaces}. In the development version of the
software version, it is possible to save the current software settings
and then open it, as we can see in Figure \ref
{fig:workspace_grass}. However, there is no option to start a specific
workspace from the command line or to start GRASS GIS from File
Manager using the file association of the workspace file (.gxw).

\vspace{0.3cm}
\begin{figure}[hbt!] 
\begin{center}
\includegraphics[width=12cm]{../pictures/workspace_grass.png} 
\caption[Management of workspaces in the Layer Manager]{Management of workspaces in the Layer Manager (Source: Personal collection)}
\label{fig:workspace_grass}
\end{center}
\end{figure}

\newpage
\vspace*{-1cm}
\subsection{Thesis objectives}
\label{sec:objectives}

The aim of the master thesis is to evaluate the changes made in the
GSoC using questionnaires, propose the final solution of the GRASS
startup mechanism and especially propose and implement a new solution
that will help enhance a first-time user experience in GRASS. There
are two fundamental questions related to the mentioned proposals,
which we will try to answer in the work:

\begin{enumerate}

\item  \noindent \textbf{How to enhance first-time user experience?}

  \noindent As already mentioned, data hierarchy in GRASS GIS can
  cause significant problems for newcomers. Therefore, the aim of this
  work is to propose the solution on how to enhance the first-time
  user experience (see subsection \ref{sec:proposal1}) and implement
  it.


\item \noindent \textbf{How to improve GRASS GIS startup mechanism?}

  As mentioned in the previous subchapter, in the special situation
  where the last opened mapset is not in the usable state, the ``old''
  startup screen still appears. This situation is rather a lack of a
  new solution introduced after GSoC and the important goal of this
  work is to propose the solution for how this shortcoming could be
  remedied (see subsection \ref{sec:proposal2}).

\end{enumerate}

\noindent Already during GSoC, it was decided that some form of
special mode for new users will need to be implemented. At the same
time, however, some questions remained unanswered regarding the
general startup mechanism and further direction of the GRASS GUI.

Therefore, the first part of the first survey entitled \textbf {Help
  improve GRASS GIS startup mechanism and Data Catalog}, see Appendix
\ref{appendix:A}, does not deal only with the second issue mentioned,
it is much broader. The goals also are to determine the level of user
satisfaction/dissatisfaction with the new solution and to obtain
general user preferences regarding further improvements of the GRASS
GUI.

As it was not clear how to grab the new special mode for first-time
users, the first survey was extended by a second thematically
different section called \textbf{Help create a better first-time user
  experience in GRASS GIS}, see Appendix \ref{appendix:B}. The answers
in this part of the survey are important for the subsequent decision
of what information the special mode for first-time users should
contain and which implementation form to choose.

The aim of the second survey called \textbf{Help improve the special
  mode for first-time users} is to find out whether users like the
proposed mockups of the Info Bar and give them space to share their
own ideas so that the implemented solution in this work is the best
possible from an objective point of view (see Appendix
\ref{appendix:C}). Both the second part of the first survey and the
second survey deal with the first question which prevails in this work
since all the performed implementations are based on it.

%% -------<<< Chapter 4: Usability testing methods >>>-------\\%%%%%%%%%%%%%%%%%%%%%%%%%%%%%%%%%%%%
\newpage
\vspace*{-1cm}
\fancyhead[RE, RO]{\fancyplain{}{\small \sl{Enhancing the first-time user experience in other software}}}
\section{Enhancing the first-time user experience in other software}
\label{sec:usability_testing}

sem bude chtít něco šoupnout

\bigskip

\noindent \textbf {QGIS 3.14}

\noindent This software is created similarly to GRASS GIS under the
auspices of OSGeo. The big change from QGIS 2 is that it includes
native support for 3D visualizations. The advantage is also the
possibility of using and creating tailor-made plugins. QGIS also
allows the use of analytical functions from GRASS, SAGA GIS software
and the GDAL library.

At the first start, Welcome to QGIS window appears, in which we can
link to the QGIS website and check the news in the given version. The
window runs in the default language of the operating system.

\vspace{0.3cm}
\begin{figure}[hbt!] 
\begin{center}
\includegraphics[width=8cm]{../pictures/qgis_startup_window.JPG} 
\caption[Welcome to QGIS 3.14 window]{Welcome to QGIS 3.14 window (Source: Personal collection)}
\label{fig:qgis_startup_window}
\end{center}
\end{figure}

\noindent After this introduction, we are redirected to the main
software window (Fig. \ref{fig:qgis_first_window}) informing about
community events and significant improvements that have taken
place. Here we can choose an empty project template or we may open
sample datasets (North Carolina, South Dakota, Alaska) provided along
with instalation.

In QGIS 3, we distinguish between the layer's and project's CRS. It is
worth mentioning that QGIS supports about 2700 known coordinate
systems, whose definitions are stored in the SQLite database installed
together with QGIS. The empty project has always global default
projection EPSG: 4326 - WGS 84.

\vspace{0.3cm}
\begin{figure}[hbt!] 
\begin{center}
\includegraphics[width=17cm]{../pictures/qgis_first_window.JPG} 
\caption[The main software window after first opening QGIS 3.14]{The main software window after first opening QGIS 3.14 (Source: Personal collection)}
\label{fig:qgis_first_window}
\end{center}
\end{figure}

\noindent In QGIS, On-the-fly transformation is enabled by default,
meaning that whenever you use layers with different coordinate systems
QGIS transparently reprojects them to the project CRS. When importing
the first layer (in this case US tracts in ESRI: 102003 system), the
dialog box shown in Fig. \ref{fig:qgis_transformation} specifies what
type of transformation to perform when changing the project CRS. These
Datum transformations are then listed in Project Properties - CRS
settings (Fig. \ref{fig:qgis_trans}). It is important to note that the
first layer added defines the coordinate system of the project. In our
case, it was changed to the US Contiguous Albers Equal Area Conic
system. In the dialog \ref{fig:qgis_trans} we can set the Predefined
coordinate system, which is related to the project, not to the layer.

Whenever a more accurate transformation is available, but is not
currently usable, QGIS 3 shows an informative warning message (see
Fig. \ref{fig:qgis_warning_window}) advising to use more accurate
transformation. Those messages appear quite often in a variety of
situations, which can definitely help new as well as experienced QGIS
users.

\begin{figure}[hbt!] 
\begin{center}
\includegraphics[width=17cm]{../pictures/qgis_warning_window.JPG} 
\caption[Informative warning message  in QGIS 3.14]{Informative warning message  in QGIS 3.14 (Source: Personal collection)}
\label{fig:qgis_warning_window}
\end{center}
\end{figure}

\begin{figure}[hbt!] 
\begin{center}
\includegraphics[width=14cm]{../pictures/qgis_transformation.JPG} 
\caption[Select the type of transformation in QGIS 3.14]{Select the type of transformation in QGIS 3.14 (Source: Personal collection)}
\label{fig:qgis_transformation}
\end{center}
\end{figure}

\begin{figure}[hbt!] 
\begin{center}
\includegraphics[width=14cm]{../pictures/qgis_trans.JPG} 
\caption[Predefined CRS and Datum Transformations in QGIS 3.14]{Predefined CRS and Datum Transformations in QGIS 3.14 (Source: Personal collection)}
\label{fig:qgis_trans}
\end{center}
\end{figure}


\newpage
\noindent \textbf{Zoner Photo Studio X} \\

\noindent After the initial login and launching the main software
window, a short first run wizard will appear and can be skipped. The
guide consists of four parts. The first section introduces the main
tabs on the left side of the software window called the Navigator and
also explains the Catalog tab, which provides quick access to
photos. The following are two pages of a guide describing photo
thumbnails and zooming. The last page of the wizard lists the right
toolbar and the three individual modules - Manager, Develop and
Editor. The main software components described in the wizard are
always marked with a blue frame and the individual information windows
are assigned to a particular part by arrows. The wizard is not only at
the beginning, but also when using the above-mentioned modules for the
first time.

Images are managed in the Catalog, which has a tree structure. We edit
the image, save it, and if we turn off the Zoner software and run it a
second time, it starts up in the last opened file. If the last used
file is deleted, the software is launched in the Manager tab in the
last opened folder, where another image can be selected for editing or
we can simply click to another folder and open or create another
image. There is a native image format of Zoner Photo Studio X with
extension * .zps, however, firstly, traditional formats such as *
.jpg, * .png, * .tif, * .gif etc. are offered to us.

\vspace{0.3cm}
\begin{figure}[hbt!] 
\begin{center}
\includegraphics[width=15cm]{../pictures/zoner.png} 
\caption[First run wizard in Zoner Photo Studio X]{First run wizard in Zoner Photo Studio X}
\label{fig:zoner}
\end{center}
\end{figure}



\newpage
\vspace*{-1cm}
\fancyhead[RE, RO]{\fancyplain{}{\small \sl{Usability testing methods}}}
\section{Usability testing methods}
\label{sec:usability_testing}

\noindent As Ana Amélia describes in her work \cite{amelia}, the
concept of usability is related to the field of
Human-Computer-Interaction (HCI). Here we can find several definitions
of what usability means. In the case of websites and software
applications, usability refers to whether or not users can achieve
specific goals with efficiency, effectiveness, and satisfaction
\cite{dishman}. As pointed out by \cite{hotjar}, the main goal of
usability testing is to test and validate the product hypothesis and
specific design decisions using the end-user perspective. We can come
across various options for dividing usability testing methods. One
possible division is nicely captured in Figure
\ref{fig:usability_testing_methods}.

\vspace{0.3cm}
\begin{figure}[hbt!] 
\begin{center}
\includegraphics[width=13cm]{../pictures/usability_testing_methods.png} 
\caption[Usability testing methods]{Usability testing methods (source: \cite{hotjar})}
\label{fig:usability_testing_methods}
\end{center}
\end{figure}

\noindent \textbf {Guerrilla testing}

\noindent This is the simplest form of Moderated + in-person usability
test best in the early stages of the product development process. The
people are asked to perform a quick usability test, often in exchange
for a small gift. Test subjects are chosen at random from a public
place, so that they may have no history with a product. This is the
reason why Guerrilla testing is not suitable for testing products that
require having special skills.

\smallskip

\noindent \textbf {Lab usability testing}

\noindent This type of Moderated + in-person usability research
usually takes place inside a controlled environment that is different
from the user’s real environment. Lab usability testing works best if
we need very comprehensive and detailed information about how the user
works with the program and what problems they encounter. Participants
perform tasks and the researcher monitors them and asks questions. It
is important that the moderator is trained and able to help
participants, but at the same time, he should let them think and not
tell them exactly what to do. After testing, it is also crucial to
discuss and analyze the specific problems the participants have faced.

\smallskip

\noindent \textbf {Phone interviews and Card sorting}

\noindent In a phone usability test, a moderator verbally instructs
participants to complete tasks on their computer and collects feedback
while the user interaction is recorded remotely. This is a very good
option to test users all over the world. Card sorting is a simple
method that involves placing concepts or features on virtual cards and
allowing participants to manipulate the cards into groups and
categories. After they sort the cards, they explain their logic to a
moderator.

\smallskip

\noindent \textbf {Contextual inquiry}

\noindent Sometimes also called the Interview/Observation is the
Unmoderated + in-person usability method especially suitable for
obtaining information about the user's habits and preferences or to
evaluate whether the user is satisfied with the product. A researcher
first asks a series of questions about the experience with the product
and then gives the user a task to work on independently. A researcher
will not provide any opinions and can only interfere if the
participant gets stuck on something. Otherwise, an observer remains
silent, focuses mainly on the emotions and behavior of the user, and
writes notes which will be then summarized in a detailed test report.

\smallskip

\noindent \textbf {Eye-tracking}

\noindent This special method allows scientists to observe the
movements of the user's eyes using a special device located on the
monitor, and to create heatmaps (where the user most often
looked). Eye-tracking requires a lab with special equipment and
software.

\smallskip

\noindent \textbf {Unmoderated + remote usability testing methods}

\noindent Test participants are asked to complete tasks alone in their
environment using their own devices. It does lead to the natural
participant's behavior, however, this type of testing is less
detailed. The main point is that a researcher needs to ensure that
every test instruction is clear. Unclear tasks can cause results that
miss the right objectives. It is not recommended to be used as a first
usability testing method since it does not go deep into the user’s
thinking. The most commonly used online testing tools are a 5-second
test and unmoderated card sorting. In the 5-second test, participants
have five seconds to look at a screenshot of the page before they
answer the question. Card sorting, described above, can also be
conducted in an unmoderated and remote manner if a researcher leaves
out the opportunity for follow-up questions.

%% -------<<< Chapter 5: Questionnaires >>>-------\\%%%%%%%%%%%%%%%%%%%%%%%%%%%%%%%%%%%%
\newpage
\vspace*{-1cm}
\fancyhead[RE, RO]{\fancyplain{}{\small \sl{Questionnaires}}}
\section{Questionnaires}
\label{sec:questionnaires}

\noindent As the reader of this work probably realized surveys
(questionnaires) were not mentioned among unmoderated + remote
usability testing. The reason is that questionnaires are not usually
considered as usability testing because it does not require to
directly test product functionality. However, in some literature
\cite{amelia} \cite{sixusability}, questionnaires are also included in
usability testing methods. In Ana Amélia's work \cite{amelia}, for
example, the survey includes both - interview and questionnaire
techniques. The questionnaire refers to a technique that can fall
under the so-called expert/heuristic method, which works with
experienced people who identify problems a less experienced user might
encounter. As the author of \cite{sixusability} writes, questionnaires
are not as numerically grounded and precise as other forms or testing,
but they can provide important feedback from the user group in a short
time. They can take the form of specific questions about the software
and its future development.

Questionnaires must, above all, be effective. Therefore, in this new
field very different from the field of programming, it was necessary
to acquire new knowledge that will help avoid beginner's mistakes. How
the surveys were conceived and what they would look like eventually
crystallized by combining two different information flows.

The first flow is based on what type of questionnaire is usually used
if we want to examine the software usability. For this purpose, we can
use the widely used standard questionnaire known as the System
Usability Scale (SUS). This questionnaire having 10 five-point items
with alternating positive or negative tone was introduced by Brooke in
1996 \cite{sus}. The standard version is shown in Figure
\ref{fig:sus}. The aim of this work is to obtain more detailed
responses from GRASS users related to a specific topic, so using this
questionnaire alone does not make much sense in this work. However,
the Slider questions (see \ref{sec:slider}) are technically realized
similarly as questions in SUS - the interviewer does not evaluate
questions but statements. It can often happen that the user would not
use the queried functionality, but they assume that the others do,
which results in a positive answer. Statements encourage the user to
better express their own opinions.

The second source that inspired the author in composing the
questionnaires was a user survey workshop within the All Things Open
platform in which the author participated. The visited video
conference\footnote{\url{https://2020.allthingsopen.org/sessions/user-experience-secrets-to-better-surveys-happier-users/}}
led by professionals Dan Zola and Kerry Thompson from SWAY UX set out
a few key rules for creating a questionnaire. According to the
lecturers, the questionnaire helps with the evaluation of customer
satisfaction as well as understanding who are users and what are their
pain points and priorities. It may be conducted not only at the start
of the new project but basically at any time the survey administrator
(developer) needs user input.

\vspace{0.3cm}
\begin{figure}[hbt!] 
\begin{center}
\includegraphics[width=14cm]{../pictures/sus.png} 
\caption[System Usability Scale Questionnaire (SUS) ]{System Usability Scale Questionnaire (SUS) (Source: \cite{sus})}
\label{fig:sus}
\end{center}
\end{figure}

\newpage
\noindent In order to understand what makes up a good questionnaire,
it is first necessary to clarify what constitutes a bad
questionnaire. The workshop summarized the following 8 errors in
particular:

\begin{itemize}
\item Too many questions (max 10 questions)
\item Convoluted questions
\item Answer choices that do not correspond with user's reasoning
\item Question that requires long and comprehensive answers (if open-ended questions, always put them at the beginning)
\item Answers that could have more than one meaning
\item Answers that do not provide any information
\item Straight line answers
\item Vague answers (scale from 1-10, where an answer is 5)
\end{itemize}

\noindent The latter vague answers can occur in the case of answers
with a rating scale. Although the authors of the workshop do not
explicitly mention the SUS questionnaire, they strongly recommend
avoiding the rating scale. The result could look similar to the
following example in Figure \ref{fig:blur_scale}. It is much more
advantageous to use either binary answers (Yes / No) or rank specific
features.

\vspace{0.3cm}
\begin{figure}[hbt!] 
\begin{center}
\includegraphics[width=12.5cm]{../pictures/blur_scale.png} 
\caption[Difficult interpretation of the rating scale answers]{Difficult interpretation of the rating scale answers (source: danzolakerrythompson)}
\label{fig:blur_scale}
\end{center}
\end{figure}

\noindent As mentioned at the beginning of the chapter, questionnaires
are not usually considered as usability testing because it does not
require to test product functionality. However, it would be very
challenging (if not impossible) in our work to ensure users having
both versions of GRASS GIS (version 7.8 and version 7.9 after GSoC)
and perform usability testing in the form of Lab usability testing or
Contextual inquiry methods. \textbf{Therefore, it was finally decided
  to use questionnaires as the main testing method.}

\subsection{Types of questions}

\noindent The author followed the advice to avoid rating scale
questions. Instead, she used Slider and Ranking methods, which
unfortunately are not offered by the free Google Forms service but
only by the paid Survey Monkey (SM) service that was eventually
used. The complete list of question types offered by the Survey Monkey
platform can be seen in Figure \ref{fig:survey_monkey_options}. Those
marked in green are used in surveys in this work. The following lines
will shed light on the advantages of the types used and the procedure
of processing them.

\vspace{0.3cm}
\begin{figure}[hbt!] 
\begin{center}
\includegraphics[width=10cm]{../pictures/survey_monkey_options.png} 
\caption[Survey Monkey answer types ]{Survey Monkey question types (Source: Personal collection)}
\label{fig:survey_monkey_options}
\end{center}
\end{figure}


\noindent \textbf {Slider}
\label{sec:slider}

\noindent This type of question, also called Visual Analog Scale
(VAS), can be used instead of a rating scale. We let respondents rate
an item or statement on a numerical scale by dragging an interactive
slider. This method is visually more pleasant than the so-called
Likert scale, which is the scale used in SUS. Besides, it best
captures the true opinion and perception of users, as there is no
limited LS 'discrete set of predetermined responses. Sliders are, for
example, widely used by healthcare professionals to determine their
patient's pain. The pain does not have discrete jumps, but it is
continuous, so it is appropriate to express it on a continuous scale
from 0 to 100. As Matevž Pesek, Alja Isakovic \cite{inproceedings}
concluded the Slider (they call it Stripe) increases the
intuitiveness, simplicity, and speed of answering the
question. Therefore, this method is widely represented in the surveys
conducted in this work. The form of the questions is inspired by the
SUS questionnaire, in which users do not essentially evaluate the
question, but try to express a degree of agreement with the opinion. A
typical example of a Slider question can be seen in Figure
\ref{fig:slider_question}.

\vspace{0.3cm}
\begin{figure}[hbt!] 
\begin{center}
\includegraphics[width=16cm]{../pictures/slider_question.png} 
\caption[A typical example of a Slider question]{A typical example of a Slider question (Source: Personal collection)}
\label{fig:slider_question}
\end{center}
\end{figure}


\newpage
\vspace*{-1cm}
\bigskip
\noindent \textbf {Ranking}

\noindent Ranking questions require the respondent to compare items to
each other by placing them in order of preference. Although ranking
answers usually have clear answers (the result cannot be an ambiguous
answer somewhere in the middle as with the rating scale), it also has
its disadvantages. It forces respondents to decide between items that
they may perceive the same. It is also essential for this type of
question to follow the general principle that the order of the
individual options is generated randomly. Otherwise, items earlier in
the list may be more likely to be ranked highest. A typical example of
a Ranking question is provided in Figure \ref{fig:ranking_question}.

\vspace{0.3cm}
\begin{figure}[hbt!] 
\begin{center}
\includegraphics[width=16cm]{../pictures/ranking_question.png} 
\caption[A typical example of a Ranking question]{A typical example of a Ranking question (Source: Personal collection)}
\label{fig:ranking_question}
\end{center}
\end{figure}

\smallskip
\vspace*{-0.5cm}
\noindent \textbf {Comment Box}

\noindent This more sophisticated technique in terms of subsequent
analyzes allows the user to share their own ideas in the form of
open-ended responses. The basis of the analysis is to determine the
topics into which the answers can be categorized. It may happen that
one answer includes more topics. In this case, it is advantageous to
divide the answers into atomic parts and then categorize them. Comment
Box questions (see Figure \ref{fig:comment_box_question}) can be
further analyzed in a similar way as Multiple Choice types - by
displaying the numbers of responses in individual topics using bar
charts.

\vspace{0.3cm}
\begin{figure}[hbt!] 
\begin{center}
\includegraphics[width=15cm]{../pictures/comment_box_question.png} 
\caption[A typical example of a Comment Box question]{A typical example of a Comment Box question (Source: Personal collection)}
\label{fig:comment_box_question}
\end{center}
\end{figure}

\newpage
\vspace*{-1cm}
\bigskip
\noindent \textbf {Multiple Choice}

\noindent For this type of question, we also distinguish whether the
respondent can choose just one option or may check more than one. In
this work, ``One option'' variant is used (see Figure
\ref{fig:multiple_choice_question}). The disadvantage of the Multiple
Choice question is the fact that we give the respondent a fixed list
of answer options, which can skew the answers. Therefore, the ``Other
(please specify)'' variant is usually added as the last place, which
allows the user to write their own answer. The analysis of a question
providing the ``Other'' variant is more demanding and requires a
similar approach as the Comment Box question since it has the
character of an open-ended question. Therefore, the responses must
first be organized into topics they deal with. If many people take the
opportunity to write their own comments, it is likely that the
responses to the question were not designed correctly. Then the
assignment of open-ended responses to those originally set is more
challenging and the telling value of responses is weakened.

\vspace{0.3cm}
\begin{figure}[hbt!] 
\begin{center}
\includegraphics[width=14cm]{../pictures/multiple_choice_question.png} 
\caption[A typical example of a Multiple Choice question]{A typical example of a Multiple Choice question (Source: Personal collection)}
\label{fig:multiple_choice_question}
\end{center}
\end{figure}

\subsection{Methods of questionnaire analysis}
\label{sec:qstat}

\noindent In this work, a statistical analysis of questions is
performed, which uses the basic methods of Explanatory Data
Analysis. Thanks to this analysis, we can discover patterns or
anomalies that occur and draw the conclusions of the surveys. It is
important to understand EDA as a process that does not have given
rules, it only depends on the analyst which of the methods to use. As
this work is purely about finding out the basic features and
characteristics of a relatively small sample of answers, which counts
a maximum of 52 respondents (the number of respondents in the first
part of the first survey), we analyze data over a single
variable/column from a dataset. Therefore, we use only basic methods
such as histograms, boxplots, and probability density functions. Other
EDA tools, for example, quantile-quantile (q-q) plots, scatter plots
or correlation matrices detect relationships between two or more
variables. Descriptive statistics, which provides us with a brief
summary of the data in the sense of Mean, Standard Deviation, and 5
elements of the box-and-whisker plot (Minimum, Maximum, 25th
percentile, median, and 75th percentile), is also often considered as
a part of EDA \cite{heroku} \cite{luminousmen}. In this work Ranking
and Slider questions are analyzed using R language using RStudio. Here
the author has applied extensive experience with this program,
especially with the Tidyverse library, which is the core of data
analysis in R and itself contains several interesting packages. In
this work, the \textit{dplyr} package is used for data
manipulation. The visualization is made through the \textit{ggplot2}
\footnote{\url{https://github.com/rstudio/cheatsheets/blob/master/data-visualization-2.1.pdf}}
package. In addition to the R language, the Python language has become
the giant of data analysis with its Pandas library in recent years.

\bigskip
\noindent \textbf{Box-and-whisker plot}

\noindent This graphical method summarizes maximum and minimum values
in data, the interquartile range, and the median. It is very practical
since all of those statistics can be seen at a glance. The graphical
representation of individual parts of the graph is captured in Figure
\ref{fig:boxplot}. The central box is enclosed by two lines
corresponding to Q1 and Q3. A line (or whisker) that extends from each
edge of the box, goes to the farthest non-outlier point in the
distribution. Outliers are points that fall more than 1.5 times the
IQR from either edge of the box. Usually, they are plotted
individually.

\vspace{0.3cm}
\begin{figure}[hbt!]
\begin{center}
\includegraphics[width=6.5cm]{../pictures/boxplot.png}
\caption[Description of box-and-whisker plot (boxplot)]{Description of box-and-whisker plot (boxplot) (Source: \cite{heroku})}
\label{fig:boxplot}
\end{center}
\end{figure}

\noindent We can identify symmetry or skewness of a distribution from
a boxplot. And if more than one boxplot is plotted on the same scale,
we can visually compare the centers, the spreads, and the extreme
values of different variables.

\bigskip
\noindent \textbf {Histogram vs. Bar Chart}

\noindent A histogram is used when working with quantitative data. It
shows the number of observations that lie in-between the range of
values, which is known as a bin. Unlike a bar chart, individual
columns touch. Categorical features cannot be visualized through
histograms. Instead, we can use bar charts where elements are taken as
individual entities, so we can e.g. rearrange the blocks, from highest
to lowest \cite{OnlineMathLearning}. Illustrative comparison of
Histogram and Bar Chart is shown in the following Figure
\ref{fig:histogram_barchart}:

\vspace{0.3cm}
\begin{figure}[hbt!]
\begin{center}
\includegraphics[width=12.5cm]{../pictures/histogram_barchart.png}
\caption[Histogram vs. Bar Chart (boxplot)]{Histogram vs. Bar Chart (Source: \cite{OnlineMathLearning})}
\label{fig:histogram_barchart}
\end{center}
\end{figure}

\bigskip
\noindent \textbf {Probability density function (PDF)}

\noindent Simply said the Probability density curve is the graphical
representation of the probability that a continuous random variable
falls in a particular class. Therefore, the total area under an entire
density curve is 100 \%. The probability that a continuous random
variable acquires a certain (exactly given) value is zero.

For a discrete random variable, we determine the probability that it
is equal to exactly some value. Such a function is called the
Probability mass function (PMF). It holds that all probabilities in
the function must be non-negative and together give the sum of 1.

\newpage
\vspace*{-1cm}
\subsubsection{Data types}

How to analyze survey responses depends on their type. We divide two
data types - qualitative and quantitative. Qualitative data can be
further divided into binary (yes / no), nominal (contains more
categories) and ordinal (also contains more categories and can be
sorted) \cite{minitab}. For example, very hot, hot, cold, very cold,
warm are all nominal data when considered individually. But when
placed on a scale and arranged in a given order (very hot, hot, warm,
cold, very cold), they are regarded as ordinal data. Regarding
descriptive statistics, we can count and determine a mode (the most
common value in a dataset) for nominal data. Measures of central
tendency for ordinal data, in which values are ranked relative to each
other but are not measured absolutely, are limited to mode or median
\cite{ordinaldata}.

Examples of nominal qualitative data are the responses to questions of
Multiple Choice type. Here, it is possible to create a bar chart
(where we can find the mode), but other descriptive statistics do not
make sense. An example of ordinal qualitative data is the responses to
questions of Ranking type. A bar chart can also be compiled for
Ranking questions, but the mean and standard deviation do not make
sense. Ordinal variables are not continuous variables and should not
be treated as if they are. Therefore, we should correctly create the
Probability mass function (PMF) for Ranking questions where we
determine a specific point probability on the y-axis that the discrete
random variable is equal to some order. However, in question Q3 in the
first part of the first survey and Q5 in the second part of the
survey, PDF charts were eventually used. Although their use is not
entirely correct, they provide better visual insight into the data
than PMF, especially if we have several PDF charts for different
variables.

Conversely, quantitative or numerical data can be characterized by a
numerical value. Quantitative variables can be further classified as
either discrete (those with a finite or countable number of possible
values) or continuous (those with an infinite or uncountable number of
possibilities) \cite{wisconsin1}. In the case of surveys conducted in
this work, the Slider questions always have the same form - we measure
the degree of agreement with the statement on a scale from 0 to
100. It is not a discrete variable, because we do not have any obvious
categories from the beginning. Those classes have to be created. In
this work, they were determined according to the SUS questionnaire:
[0, 20] - Strongly disagree, (20, 40] - Disagree, (40, 60] - Neutral,
(60, 80] - Agree and (80, 100] - Strongly agree.



%% -------<<< Chapter 7: Analysis of the results of the first survey>>>-------\\%%%%%%%%%%%%%%%%%%%%%%%%%%%%%%%%%%%%

\newpage
\vspace*{-1cm}
\fancyhead[RE, RO]{\fancyplain{}{\small \sl{Analysis of the results of the first survey}}}
\section{Analysis of the results of the first survey}
\label{sec:qstat}

\noindent The main part of this work consists of two
questionnaires. The first questionnaire was released on 23 October and
stopped on 29 October 2020. It has two separate parts. The first one
contains six questions dealing with the improvement of the GRASS GIS
startup mechanism and the assessment of the state after GSoC. The
second part of the survey which is even more important for this master
thesis contains 5 questions focusing on the enhancement of the
newcomers' experience. In other words, it focuses on possibilities of
how to enrich the existing ``demolocation'' concept so that the new
users can find their way around the software as quickly and
conveniently as possible. The implementation part of this work is
based on the second part of the first survey and then especially on
the second survey, which was released roughly a month later and is
analyzed in Section \ref{sec:qstat2}.

\subsection{Part 1: GRASS startup and Data Catalog}

\noindent The first part of the survey called \textbf {Help improve
  GRASS GIS startup mechanism and Data Catalog} returns to the changes
that were implemented within the GSoC. It tries both - to get feedback
from users and to answer questions that remain unanswered after
GSoC. From this point of view, the most fundamental question is No. 2,
which finds out the preferences regarding the way of starting GRASS
GIS in a situation where the last mapset is not in a usable
state. Questions 4, 5, and 6 are also related to the further direction
of GRASS GIS (not only in terms of the startup mechanism) while
questions 1 and 3 assess the benefits of GSoC. The first part of the
survey was attended by 52 respondents, the completion rate was high -
96 \% and no respondent skipped any questions. We can see a weekly
graph of the number of responses for a specific day in Figure
\ref{fig:survey1_part2_insight2}.

\begin{figure}[hbt!] 
\begin{center}
\includegraphics[width=10.5cm]{../surveys/analyzed_data/survey1_part1_insight2.png} 
\caption[Survey 1 Part 1: Responses by day]{Survey 1 Part 1: Responses by day (Source: Basic analyzes provided by the SM)}
\label{fig:survey1_part1_insight2}
\end{center}
\end{figure}

\newpage
\noindent \textbf{Question 1: What do you think about the following statement? \\
  The partial removal of the startup screen and improvement of the
  Data Catalog simplifies the initial introduction to the software and
  further work.}
\par\noindent\rule{\textwidth}{0.4pt}

\noindent This question has the form of a Slider where 0 means
complete disagreement with the statement and 100 means complete
agreement. If we only worked with an average value of 70.9, we would
conclude that enthusiasm probably prevails, but it is not so
certain. The box-whisker-plot in Figure
\ref{fig:survey1_part1_question1_boxplot} shows that the median is
significantly higher - 78.5 points out of a 100. If we look at the
histogram in Figure \ref{fig:survey1_part1_question1_histogram}, which
was divided into 5 parts exactly according to the number of bins in
the original SUS, we can see that in the last interval ``Strongly
agree'' there are 22 responses out of 52. On this basis, we can
conclude then that the vast majority like the situation after GSoC,
however, there is also a minority of negative opinions.

\begin{figure}[hbt!] 
\begin{center}
\includegraphics[width=12cm]{../surveys/analyzed_data/survey1_part1_question1_excel_histogram.png} 
\caption[Survey 1 Part 1 Question 1: Histogram]{Survey 1 Part 1 Question 1: Histogram (Source: Personal collection)}
\label{fig:survey1_part1_question1_histogram}
\end{center}
\end{figure}

\vspace{0.3cm}
\begin{figure}[hbt!] 
\begin{center}
\includegraphics[width=10cm]{../surveys/analyzed_data/survey1_part1_question1_excel_boxplot.png} 
\caption[Survey 1 Part 1 Question 1: Boxplot]{Survey 1 Part 1 Question 1: Boxplot (Source: Personal collection)}
\label{fig:survey1_part1_question1_boxplot}
\end{center}
\end{figure}

\newpage
\noindent \textbf{Question 2: How do you think GRASS should start when the last mapset is not in a usable state (was deleted or is in use)?}
\par\noindent\rule{\textwidth}{0.4pt}

\noindent The result of this question put in the Multiple Choice form
with the possibility of open-ended responses is very unexpected and
unclear. Some open-ended answers strongly suggest that respondents did
not understand from the previous context to the question (see Appendix
\ref{appendix:A}, page 2) that the startup screen will not be visible
at all in other cases. Another drawback is the only two close-ended
choices. Most people, therefore, preferred something familiar (startup
screen) to the new concept of Demolocation.

One of the ways to proceed here is not to give to the result of this
question in terms of the bar chart in Figure
\ref{fig:survey1_part1_question2_histogram_sm}, as the possibilities
of answers to this question were not well-conceived, and focus mainly
on open-ended responses. Ten respondents choose neither of the two
questions offered and took the opportunity to write their own
proposal. Of the ten answers, which are categorized in the table in
Figure \ref{fig:survey1_part1_question2_other_answers}, only one
respondent \#31 is for maintaining the original start screen. There
are some ideas of displaying a simple dialog in the form of a warning
message, which offers a user some other options - e.g. to open a
demolocation, an existing mapset, or to create a new mapset. There was
also a suggestion that it would not be a dialog, but only a pop-up
message.

\vspace{0.3cm}
\begin{figure}[hbt!] 
\begin{center}
\includegraphics[width=17cm]{../surveys/analyzed_data/survey1_part1_question2_histogram_sm.png} 
\caption[Survey 1 Part 1 Question 2: Bar Chart]{Survey 1 Part 1 Question 2: Bar Chart (Source: Basic analyzes provided by the SM)}
\label{fig:survey1_part1_question2_histogram_sm}
\end{center}
\end{figure}

\vspace{0.3cm}
\begin{figure}[hbt!] 
\begin{center}
\includegraphics[width=15cm]{../surveys/analyzed_data/survey1_part1_question2_other_answers.pdf} 
\caption[Survey 1 Part 1 Question 2: Classification of open-ended responses]{Survey 1 Part 1 Question 2: Classification of open-ended responses (Source: Personal collection)}
\label{fig:survey1_part1_question2_other_answers}
\end{center}
\end{figure}

\noindent The first option, therefore, how this case could be solved
would be to use the Info Bar, which would not have an informative
character (as with the proposed solution for first-time users in the
second questionnaire), but would have the character of a warning. This
Warning Info Bar explains why the user was redirected to Demolocation
and instructs the user to open their own project.

The second solution is to create a simple startup screen, very similar
to Moritz Lennert's Proposal
B1\footnote{\url{https://trac.osgeo.org/grass/wiki/wxGUIDevelopment/New\_Startup\#Changeparadigm}},
which explains the situation to the user (the last mapset used has
been deleted or is in use by another process) and suggests further
steps.

The advantage of changes made after GSoC is that changing the
database, location or mapset is very simple through the new Data
Catalog, as well as saving and opening workspaces. Therefore, not only
the author but also other developers are inclined rather to the
variant to completely remove any form of startup screen and employ
Info Bars instead.

\par\noindent\rule{\textwidth}{0.4pt} \\
\noindent \textbf{Question 3: Please, rank how useful these features in Data Catalog would be (or already are) for you (1 = the most useful).}
\par\noindent\rule{\textwidth}{0.4pt}

\noindent This question asks the GRASS user to evaluate the benefits
of the new features that were introduced after GSoC (see Figure
\ref{fig:function}) . The implementation of \textit{Small icons
  distinctive mapping, locations, GRASS databases, and layers (vector,
  raster)} is the work of Anna Petrasova, other functions are part of
the implementations performed by the author. The evaluation of
functions is performed by sorting them from the most useful (number 1)
to the least useful (number 7).

According to the means in Figure
\ref{fig:survey1_part1_question3_descriptive_stats_sm} the greatest
success have \textit{New management icons}. However, the use of mean
and standard deviation is quite misleading for ordinal types of
variables. It is better to focus on the median whose values are in
Figure \ref{fig:survey1_part1_question3_descriptive_stats_sm} marked
in a red box. The median is the smallest for \textit{Small icons
  distinguishing mapsets, locations, GRASS databases, and layers
  (vector, raster)}. Although the responses do not have the character
of a continuous random variable and the Figure
\ref{fig:survey1_part1_question3_pdf} is somewhat misleading in terms
of statistics, \textit{Small icons distinguishing mapsets, locations,
  GRASS databases, and layers (vector, raster) } are also the most
successful here. In the second place, there are \textit{New management
  icons} and in third place, we can find \textit{Creating, renaming
  and deleting mapset or location}.

Interestingly, no special success was achieved by \textit{Adding
  multiple GRASS databases}, which were relatively difficult to
implement. As can be seen from the stacked bar chart in Figure
\ref{fig:survey1_part1_question3_histogram} and also the boxplot in
Figure \ref{fig:survey1_part1_question3_boxplot_r}, the most
controversial is \textit{Mapset access info (current, in use, and a
  different user)}, which some users rated as the most useful and a
similar number of users as the least useful.

\vspace{0.3cm}
\begin{figure}[hbt!] 
\begin{center}
\includegraphics[width=17cm]{../surveys/analyzed_data/survey1_part1_question3_descriptive_stats_sm.png} 
\caption[Survey 1 Part 1 Question 3: Descriptive statistics]{Survey 1 Part 1 Question 3: Descriptive statistics (Source: Basic analyzes provided by the SM)}
\label{fig:survey1_part1_question3_descriptive_stats_sm}
\end{center}
\end{figure}

\vspace{0.3cm}
\begin{figure}[hbt!] 
\begin{center}
\includegraphics[width=17cm]{../surveys/analyzed_data/survey1_part1_question3_histogram.png} 
\caption[Survey 1 Part 1 Question 3: Stacked Bar Chart ]{Survey 1 Part 1 Question 3: Stacked Bar Chart (Source: Basic analyzes provided by the SM)}
\label{fig:survey1_part1_question3_histogram}
\end{center}
\end{figure}

\newpage
\vspace{0.3cm}
\begin{figure}[hbt!] 
\begin{center}
\includegraphics[width=15.5cm]{../surveys/analyzed_data/survey1_part1_question3_boxplot_r.png} 
\caption[Survey 1 Part 1 Question 3: Boxplot]{Survey 1 Part 1 Question 3: Boxplot (Source: Personal collection - R analysis)}
\label{fig:survey1_part1_question3_boxplot_r}
\end{center}
\end{figure}

\vspace{0.3cm}
\begin{figure}[hbt!] 
\begin{center}
\includegraphics[width=15.5cm]{../surveys/analyzed_data/survey1_part1_question3_pdf.png} 
\caption[Survey 1 Part 1 Question 3: Probability Density Function]{Survey 1 Part 1 Question 3: Probability Density Function (Source: Personal collection - R analysis)}
\label{fig:survey1_part1_question3_pdf}
\end{center}
\end{figure}

\newpage
\noindent \textbf{Question 4: Which features would you like to add?}
\par\noindent\rule{\textwidth}{0.4pt}

\noindent From the point of view of the Data Catalog, in Figure
\ref{fig:survey1_part1_question4_open_ended}, we can see proposals for
the EPSG code shown after the location name, cloning the location,
deleting multiple layers via the context menu, displaying space-time
datasets (STDS), or displaying saved workspaces. Regarding things
unrelated to the Data Catalog, three users would like an easier and
clearer way to add WMS/WFS using a new icon.

\begin{figure}[hbt!] 
\begin{center}
\includegraphics[width=15.5cm]{../surveys/analyzed_data/survey1_part1_question4_open_ended.png} 
\caption[Survey 1 Part 1 Question 4: Classification of open-ended answers]{Survey 1 Part 1 Question 4: Classification of open-ended answers (Source: Personal collection)}
\label{fig:survey1_part1_question4_open_ended}
\end{center}
\end{figure}

\newpage
\noindent \textbf{Question 5: Because we have limited screen space, we need to think about where we can add new features.Where would you add them?}
\par\noindent\rule{\textwidth}{0.4pt}

\noindent In this Multiple Choice question analyzed in Figure
\ref{fig:survey1_part1_question5_descriptive_stats_sm} using the bar
chart, most respondents agree to add additional functions to the
context menu. Interestingly, 13.5 \% of respondents think that
\textit{no additional features should be added, there is little space
  for them}. That is almost a seventh of the respondents, so also a
relatively significant part. So there is a certain fear that the added
functions may rather reduce the clarity of the current solution. This
is also evidenced by the fact that in the Q4 none of the respondents
mention adding more complex functions to the Data Catalog, such as
Data Import. The improvements mainly concern the management of data
hierarchy in GRASS.

\vspace{0.3cm}
\begin{figure}[hbt!] 
\begin{center}
\includegraphics[width=17cm]{../surveys/analyzed_data/survey1_part1_question5_descriptive_stats_sm.png} 
\caption[Survey 1 Part 1 Question 5: Bar Chart]{Survey 1 Part 1 Question 5: Bar Chart (Source: Basic analyzes provided by the SM)}
\label{fig:survey1_part1_question5_descriptive_stats_sm}
\end{center}
\end{figure}

\newpage
\noindent \textbf{Question 6: So, what do you think about the following statement? I would start GRASS using the file association of the workspace file (.gxw) frequently.}
\par\noindent\rule{\textwidth}{0.4pt}

\noindent This question takes the form of a Slider and was
intentionally conceived in this somewhat strict way. It depends on the
opinion of individuals whether they would really use this
functionality often, not on the general belief, which can be distorted
by the fact that in other software this functionality is a matter of
course. Among highly evaluated commercial and open-source software by
GISGeography \cite{gisgeography} (ArcGIS Pro, Geomedia Advantage,
MapInfo Professional, QGIS 3, gvSIG, GRASS GIS, ILWIS, SAGA GIS),
according to author's conclusions, GRASS GIS is the only software of
mentioned together with ILWIS, that cannot be started using the file
association of the workspace file. Nevertheless, based on the results,
users do not seem to mind.

The author can name several reasons why. After changes within GSoC,
GRASS GIS starts in the last open mapset to the Data tabs, which
allows easy switching between mapsets, opening GRASS databases, saving
and opening workspaces, etc. It is therefore very easy to work with
the data hierarchy. In addition, the data is stored directly in the
mapsets, so it is not necessary to remember where in the disk the data
is located, such as in QGIS. Another reason is that GRASS GIS is often
started from the command line.

The average value of the degree of agreement with the statement is 51,
as you can see from Figure
\ref{fig:survey1_part1_question6_boxplot}. The median is slightly
higher, just over 54 points. Both of these values are more or less
meaningless. The most interesting view is offered by the histogram in
Figure \ref{fig:survey1_part1_question6_histogram} and PDF in Figure
\ref{fig:survey1_part1_question6_r_pdf}. The largest density values
are in the range of 50 - 75, according to which we can conclude that
the answer whether to implement is rather yes, but it is not a
functionality that is perceived as essential.

\vspace{0.3cm}
\begin{figure}[hbt!] 
\begin{center}
\includegraphics[width=12.5cm]{../surveys/analyzed_data/survey1_part1_question6_boxplot.png} 
\caption[Survey 1 Part 1 Question 6: Boxplot]{Survey 1 Part 1 Question 6: Boxplot (Source: Personal collection)}
\label{fig:survey1_part1_question6_boxplot}
\end{center}
\end{figure}

\vspace{0.3cm}
\begin{figure}[hbt!] 
\begin{center}
\includegraphics[width=14cm]{../surveys/analyzed_data/survey1_part1_question6_histogram.png} 
\caption[Survey 1 Part 1 Question 6: Histogram]{Survey 1 Part 1 Question 6: Histogram (Source: Personal collection)}
\label{fig:survey1_part1_question6_histogram}
\end{center}
\end{figure}

\vspace{0.3cm}
\begin{figure}[hbt!] 
\begin{center}
\includegraphics[width=15cm]{../surveys/analyzed_data/survey1_part1_question6_r_pdf.png} 
\caption[Survey 1 Part 1 Question 6: Probability Density Function]{Survey 1 Part 1 Question 6: Probability Density Function (Source: Personal collection - R analysis)}
\label{fig:survey1_part1_question6_r_pdf}
\end{center}
\end{figure}


\newpage
\vspace*{-1cm}
\subsection{Part 2: Better first-time user experience in GRASS}

\noindent The second part of the survey called \textbf{Help create a
  better first-time user experience in GRASS GIS} seeks to get user
preferences on how they would like to improve the first-time user
experience. This survey consists mainly of Multiple Choice and Comment
Box types of questions, so the evaluation is very subjective and there
are exceptional answers for which the author was not entirely sure
whether she understood them correctly. This is, after all, one of the
disadvantages of questionnaires and remote usability testing in
general.

This part of the survey offers two ways to improve the first-time user
experience that the author has noticed with other software - Info Bars
(QGIS 3) and First Run Wizard (Zoner Photo Studio X). Respondents
evaluate these topics in the first two questions, thus giving feedback
on which of the above options they would prefer in GRASS. Question 3
is very open and users write their own suggestions on how to improve
the first-time user experience. In Question 4, users share ideas for
software that is user-friendly, while in Question 5, we then find out
which specific things cause problems for users and what any first-time
advice should be about. The second part of the first survey was
attended by 46 respondents, the completion rate was high - 97
\%. Question 4, which was optional, was skipped 14 times. Other
questions were answered by all participants, however, the answers were
not always relevant, so not all are part of the analyzes. We can see a
weekly graph of the number of responses for a specific day in Figure
\ref{fig:survey1_part2_insight2}.

\vspace{0.3cm}
\begin{figure}[hbt!] 
\begin{center}
\includegraphics[width=15cm]{../surveys/analyzed_data/survey1_part2_insight2.png} 
\caption[Survey 1 Part 2: Responses by day]{Survey 1 Part 2: Responses by day (Source: Basic analyzes provided by the SM)}
\label{fig:survey1_part2_insight2}
\end{center}
\end{figure}

\newpage
\noindent \textbf{Question 1: Do you like the idea of First Run Wizard (inspired by Zoner implementation)?}
\par\noindent\rule{\textwidth}{0.4pt}

\noindent Users mostly like the idea of First Run Wizard, but there
are also a lot of comments in Figure
\ref{fig:survey1_part2_question1_all} highlighted in pink, which First
Run Wizard finds rather annoying. After all, comments \#1 and \#2 also
mean ``No, because...'' rather than ``Yes, but...''. If we then
combine the subgroups into two large groups ``Yes'' and ``No'', we
will come to the conclusion that 33 (71.7 \%) respondents are for and
the remaining 13 (28.3 \%) respondents against First Run Wizard.

\vspace{0.3cm}
\begin{figure}[hbt!] 
\begin{center}
\includegraphics[width=15cm]{../surveys/analyzed_data/survey1_part2_question1_all.png} 
\caption[Survey 1 Part 2 Question 1: Bar Chart, descriptive statistics and open-ended responses]{Survey 1 Part 2 Question 1: Bar Chart, descriptive statistics and open-ended responses (Source: Basic analyzes provided by the SM)}
\label{fig:survey1_part2_question1_all}
\end{center}
\end{figure}

\noindent The comment \#2 recommends creating pop-ups that themselves
contain only the most important information, but refer to a written
and video tutorial. So, the point is to include as few distractions as
possible in the software itself, but to refer well to detailed
information, for example in the form of a ``Learn more'' button.

\par\noindent\rule{\textwidth}{0.4pt}
\noindent \textbf{Question 2: Do you like the idea of first-time mode info bars (visually similar to info bars in QGIS implementation)?}
\par\noindent\rule{\textwidth}{0.4pt}

\noindent People largely like both the Info Bars and the First Run Wizard. However, there are fewer ``Yes, but...'' comments in Info Bars, and unlike the previous question, they are literally ``Yes, but...'', as we can see in Figure \ref{fig:survey1_part2_question2_all}. 

\vspace{0.3cm}
\begin{figure}[hbt!] 
\begin{center}
\includegraphics[width=15.5cm]{../surveys/analyzed_data/survey1_part2_question2_all.png} 
\caption[Survey 1 Part 2 Question 2: Bar Chart descriptive statistics and classification of open-ended responses]{Survey 1 Part 2 Question 2: Bar Chart, descriptive statistics and classification of  open-ended responses (Source: Basic analyzes provided by the SM)}
\label{fig:survey1_part2_question2_all}
\end{center}
\end{figure}

\noindent A relatively large proportion of respondents (17.4 \%) think
that these icons will be ignored by new GRASS users. If we combine the
subgroups into two main groups ``Yes'' and ``No'', we conclude that 34
(73.9 \%) respondents are for and 12 (26.1 \%) against Info Bars. This
is a slightly better balance than in Question 1. But even here we must
be careful. The difficult task will be to find a compromise between
the fact that the information icons must be placed in the right place
so that the user can ignore them as little as possible, but at the
same time must not act as a warning, which is also pointed out by
comment \#3.

\par\noindent\rule{\textwidth}{0.4pt}
\noindent \textbf{Question 3: Do you have other ideas that would lead you to more straightforward navigation in the software?}
\par\noindent\rule{\textwidth}{0.4pt}
\noindent For this question, GRASS GIS users were very shared, which
resulted in a very detailed analysis of the answers into 12
groups. However, the variety of responses is so wide that many
responses could not be included, so they ended up in a separate
``Other'' group. In the following lines, the author summarizes and
discusses in more detail several opinions that were expressed. The
color representing categories in Figures
\ref{fig:survey1_part2_question3_open_ended-1_1},
\ref{fig:survey1_part2_question3_open_ended-2_1},
\ref{fig:survey1_part2_question3_open_ended3_1} is purely random.

The first topic that permeates the whole questionnaire is how to
better explain the GRASS GIS data hierarchy to newcomers. In this
question, this topic is mentioned by respondents \#2, \#8, \#19,
\#22. For example, respondent \#8 suggests a ``First Time Help''
pop-up window that explains folders on a disk, locations,
etc. However, the topic of GRASS data hierarchy appears in other
questions as well. For example, in Question 4 respondent \#1 talks
about the old concept of location. However, as can be seen from the
answers \#25 and \#27, some users are satisfied with the current
system.

Either way, data hierarchy in GRASS is usually one of the main topics
of video calls of the developer community (link to the attached video
call) and the only agreement is that the concept of
\textit{database/location/mapset} is abstruse to new users. The truth
is that if database/location/mapset were named differently and more
intuitively (for instance \textit{database/project/subproject}) and
thus closer to the standard of other open-source software, then the
word ``old'' would not be part of the criticism. Therefore, perhaps
the most feasible proposal that will not interfere so much with the
implementation of GRASS is to maintain the concept but to change the
terminology.

The second important topic is better documentation, for example with
the use of videos. Respondent \#26 would even like very detailed PDF
manuals with print screens and information on each button and
functionality.  It should be noted here that the proposed solutions
that will improve the first-time user experience must be sustainable
also in terms of further development, which will probably be crucial
in the future, as the community around GRASS is very lively.
\newpage
\vspace{0.3cm}
\begin{figure}[hbt!] 
\begin{center}
\includegraphics[width=17cm]{../surveys/analyzed_data/survey1_part2_question3_open_ended-2_2.png} 
\caption[Survey 1 Part 2 Question 3: Classification of open-ended responses - part 1]{Survey 1 Part 2 Question 3: Classification of open-ended responses - part 1 (Source: Personal collection)}
\label{fig:survey1_part2_question3_open_ended-1_1}
\end{center}
\end{figure}

\newpage
\vspace{0.3cm}
\begin{figure}[hbt!] 
\begin{center}
\includegraphics[width=16cm]{../surveys/analyzed_data/survey1_part2_question3_open_ended-2_3} 
\caption[Survey 1 Part 2 Question 3: Classification of open-ended responses - part 2]{Survey 1 Part 2 Question 3: Classification of open-ended responses - part 2 (Source: Personal collection)}
\label{fig:survey1_part2_question3_open_ended-2_1}
\end{center}
\end{figure}

\newpage
\vspace{0.3cm}
\begin{figure}[hbt!] 
\begin{center}
\includegraphics[width=15cm]{../surveys/analyzed_data/survey1_part2_question3_open_ended-2_1} 
\caption[Survey 1 Part 2 Question 3: Classification of open-ended responses - part 3]{Survey 1 Part 2 Question 3: Classification of open-ended responses - part 3 (Source: Personal collection)}
\label{fig:survey1_part2_question3_open_ended3_1}
\end{center}
\end{figure}

\noindent Therefore, it is advantageous to focus on smaller outputs,
which is easy to edit in case of changes, rather than doing
``inflexible'' tutorials, whether in the form of PDFs or videos, which
can be very outdated in a short time, thus for new users rather
confusing. However, this does not mean that short videos or clearer
documentation could not be created. Inspiration can come here from
QGIS, as suggested by respondent \#14 in question 5. Respondent \#7
also encounters the relationship between QGIS and GRASS. As mentioned
in \ref{subsection:GIS software} GRASS modules can also be run through
QGIS. However, the GRASS function in QGIS has a different description
than the same module in GRASS. This should be unified on the GRASS
side.

From the point of view of the further direction of this master thesis,
a very important topic is how to improve the first-time user
experience directly in the software, in other words how to improve
demolocation where GRASS starts automatically after the first
start. The answer \#3 is closely related to this since it asks for
immediate display of the map. After all, for example, the open-source
software Blender for modeling and rendering 3D computer graphics shows
the cube when started. It is therefore natural for a new GRASS user to
see the map. Furthermore, the classification also shows that brief
advice on how to start would be valuable for newcomers and six
respondents (\#5, \#8, \#9, \#12, \#15, \#18) mentions some form of
info icons. Respondents \#5 and \#12 also point out the importance of
the initial data and its easy import into the GRASS GIS. Two opinions
call for putting GRASS into a single window.

The answers also point out some of the problems that users face
without explicitly mentioning them. Respondents \#10 and \#11 want a
``Tool search'' which is already part of the Modules tab. This may
indicate that users did not understand the meaning of other tabs
(probably did not notice them). By the way, Question 5 also draws
attention to this problem, where 9 respondents out of 46 classified
\textit{Description of main tabs (Data, Display, Modules, Console,
  Python] and Map Display} as the advice that would help them most in
their initial orientation in the software.

\par\noindent\rule{\textwidth}{0.4pt}
\noindent \textbf{Question 4: What software do you think does a good job of providing a good first-time user experience? (optional)}
\par\noindent\rule{\textwidth}{0.4pt}
\noindent In this optional question, QGIS is mentioned most
often. However, the solution in GRASS may not be the same as in
QGIS. Ideally it will be the best solution for GRASS, see \#42 in Q3.

\vspace{0.3cm}
\begin{figure}[hbt!] 
\begin{center}
\includegraphics[width=17cm]{../surveys/analyzed_data/survey1_part2_question4_open_ended_1.png} 
\caption[Survey 1 Part 2 Question 4: Classification of open-ended responses]{Survey 1 Part 2 Question 4: Classification of open-ended responses (Source: Personal collection)}
\label{fig:survey1_part2_question4_open_ended_1}
\end{center}
\end{figure}

\newpage
\noindent \textbf{Question 5: Let's imagine you are a first-time
  user. What would help you significantly in your initial orientation
  in the software? Please, rank those features according to the
  importance (1 = the most important).}
\par\noindent\rule{\textwidth}{0.4pt}

\noindent This question, where users sorted different features
according to how much they would help them in their initial
orientation, has no clear answers at all. Although the
\textit{Description of main tabs (Data, Display, Modules, Console,
  Python, and Map Display} variant has the lowest preference according
to the median (see Figure \ref{fig:survey1_part2_question5_stats}) it
appears very often on the first place as captured by the bar chart in
Figure \ref{fig:survey1_part2_question5_histogram_r}. The answers
suggest that the software should provide advice on all of these
selected aspects because more or less, all of these aspects are
problematic for new users. The question remains what advice and in
what form to include directly in the demolocation and what advice
should no longer be included, but it should be well-referred to.
    
\vspace{0.3cm}
\begin{figure}[hbt!] 
\begin{center}
\includegraphics[width=16cm]{../surveys/analyzed_data/survey1_part2_question5_stats.png} 
\caption[Survey1 Part 2 Question 5: Descriptive statistics]{Survey1 Part 2 Question 5: Descriptive statistics (Source: Personal collection - R analysis)}
\label{fig:survey1_part2_question5_stats}
\end{center}
\end{figure}

\vspace{0.3cm}
\begin{figure}[hbt!] 
\begin{center}
\includegraphics[width=15cm]{../surveys/analyzed_data/survey1_part2_question5_histogram_r.png} 
\caption[Survey1 Part 2 Question 5: Stacked Bar Chart]{Survey1 Part 2 Question 5: Stacked Bar Chart (Source: Personal collection - R analysis)}
\label{fig:survey1_part2_question5_histogram_r}
\end{center}
\end{figure}

%% -------<<< Chapter 7: GRASS GIS Development Proposals>>>-------\\%%%%%%%%%%%%%%%%%%%%%%%%%%%%%%%%%%%%
\newpage
\vspace*{-1cm}
\fancyhead[RE, RO]{\fancyplain{}{\small \sl{GRASS GIS Development Proposals}}}
\section{GRASS GIS Development Proposals}
\label{sec:proposal}

\noindent As described in the subchapter \ref{sec:objectives}, this
work consists of 2 seemingly independent parts. The first part (the
main part of this work) related to the second part of the first survey
and the second survey improves the concept of default location by the
so-called first-time user mode, which aims to improve the first-time
user experience. The second part, related to the first part of the
first survey, builds significantly on the changes made in the GSoC and
addresses the shortcomings of the new startup mechanism. The first
main topic of this work focuses exclusively on new users, while the
second topic related to the complete removal of the original startup
screen (so that it does not appear even in a situation where the last
mapset is not in a usable state) is a topic that affects all existing
users.

\subsection{How to enhance first-time user experience}
\label{sec:proposal1}

After running GRASS in the version after GSoC, the new user is
redirected to the default location (demolocation) to the Data
tab. Here he/she can see a project template containing a world
map. (In the future, the default location is likely to be enriched
with additional maps.) The analysis of the second survey confirmed the
assumption that in order to make GRASS a more user-friendly tool right
from the outset, there is some form of help (whether in form of First
Run Wizard or Info Bars) a necessity. Although the Data Catalog
provides a visual idea of the hierarchical data structure in GRASS, no
clue would explain the concept of the default location, nor locations
and mapsets in general. But it's not just a misunderstanding of the
GRASS data hierarchy, as the developers initially thought. Users also
have problems with the meaning of tabs, especially with the Modules
tab. The responses to Question 5 also point out that the advice on the
data import would be useful as well.

In the most likely scenario, as a new GRASS user, we would try to
import our data, display it in the required design, and perform
analyzes. To do this, in the first step, we should create a location
in the coordinate system of our data. In the second step, depending on
the data type, we can already import vector or raster data. In the
third step, we can visit the Modules tab if want to analyze the data
straight away. However, if we only want to change the layer
properties, we should go to the Display tab.

\vspace{0.3cm}
\begin{figure}[hbt!] 
\begin{center}
\includegraphics[width=16cm]{../pictures/first-time_user_diagram.png} 
\caption[Flowchart of First-time user mode]{Flowchart of First-time user mode (Source: Personal collection)}
\label{fig:first-time_user_diagram}
\end{center}
\end{figure}

For these situations, the author has proposed the sequence of small
hints which can be seen in Figure
\ref{fig:first-time_user_diagram}. Those texts displayed in the orange
box essentially suggest a special first-time user mode, which we can
also perceive as a First Run Wizard in terms of continuity. If the
user follows the advices, he will eventually meet all three advice in
that order. So it is not possible to skip this mode as in Zoner Photo
Studio, but if we do not follow the advice, we can avoid it. Since
small hints are displayed in predefined situations, it was decided to
implement them with a similar solution that is used in QGIS 3 - in the
form of the Info Bar. After all, the Info Bar had better response than
the First Run Wizard in the second survey. Except for hints, the Info
Bar in GRASS will include a button(s) that offer the user the
necessary functions without the need to search. A more detailed
description of the first-time mode, including Info Bar mockups, is
contained in section \ref{sec:qstat2}.

\subsection{How to improve GRASS GIS startup mechanism}
\label{sec:proposal2}

In the first part of the first survey, Q4 and Q5 mainly concern the
Data Catalog and Management Icons, which development is important,
however, the implementation is beyond the scope of this work. From the
point of view of this work, the most important question is number 2,
which solves the case when GRASS wants to start in the last used
mapset, however, this mapset is not available for one of these reasons
- it has either been deleted or it is used by another
process. Unfortunately, this question was not drafted very cleverly in
terms of the answer options offered, which is evident from a large
number of open-ended responses.

At first glance, the number of respondents who would choose the
modernized version of the startup screen prevails. However, only 2 out
of 10 open-ended responses suggest some form of the startup
screen. Although the total number of respondents proposing a startup
screen is 34 (32 + 2), ie. 65 \% percent of all respondents, we can
not talk about a significant majority opinion. Besides, in open-ended
responses, ideas with some form of information or error message very
often appear, which is a very interesting variant, which the author of
the work did not think of when compiling the survey. In terms of
implementation complexity, however, it is a simple option, as the
proposed Info Bar for first-time users can cover both purposes - it
will primarily serve as a help for new users and secondarily it can
serve as an information channel for existing users. In the latter
context, the Info Bar can inform users about non-standard situations
and speed up their work (e.g. when creating a location, it will
automatically offer the creation of a mapset, etc.).

The display of the Info Bar in a non-standard situation is related to
the existing startup mechanism, which still has not completely gotten
rid of the old startup screen. If we start GRASS GIS for the first
time, we meet a default location. In other cases, GRASS tries to boot
into the last used mapset. However, there may be a problem where the
last used mapset is not in a usable state. Two suggestions on how to
solve this situation were compiled, which are presented using
Flowcharts in Figures \ref{fig:normal_user_diagram} and
\ref{fig:normal_user_diagram2}, where the second proposal is an
extension of the first proposal.

\newpage
In the first proposal in Figure \ref{fig:normal_user_diagram} we do
not consider the last used location at all. If the last used mapset is
in an unusable state, the user will always be redirected to the
default location. We do not prohibit the user from working in the
default location as in the normal location. Then, however, there is a
risk that the PERMANENT mapset can be used by another process. This
situation can occur, for example, when a user starts two GRASS
sessions. The first instance starts in the last used mapset, but the
second instance can no longer start this way.

\vspace{0.3cm}
\begin{figure}[hbt!] 
\begin{center}
\includegraphics[width=16.5cm]{../pictures/normal_user_diagram.png} 
\caption[Flowchart of GIS startup for existing users: Proposal 1]{Flowchart of GIS startup for existing users: Proposal 1 (Source: Personal collection)}
\label{fig:normal_user_diagram}
\end{center}
\end{figure}

\noindent It is therefore essential to check whether the PERMANENT
mapset in the default location is in the usable state and if it is
not, we need to create a new mapset in the default location, into
which GRASS will start in case of emergency. This mapset can be named
after the user, which is a concept proposed in the second survey as a
standard solution for first-time users. (In the end, it was not
implemented since the second survey found that the existence of a user
mapset in the standard version of the default location is rather
confusing than useful).

\newpage
When starting in the default location, the user will see the
notification in the form of the Info Bar saying the reason of start in
the default location and advising a user on what to do next in this
situation. The solution is simple as in the new version of the Data
Catalog after GSoC it is very easy to add a new database to the Data
Catalog, create a new location, etc. The Info Bar can therefore have
the character of an information icon, as in the case with help for
first-time users.

Proposal 1 can be extended with another idea. To offer the user the
most similar state to the last GRASS run, we can start GRASS in the
PERMANENT mapset of the last used location. However, it presents a new
concept, which assumes that the PERMANENT mapset is some kind of
default mapset. This would mean incorporating this idea into other
segments of GRASS. For example, in the Data Catalog context menu,
there should be a possibility to switch to location (its PERMANENT)
from the location node. In this proposal, the default location is
taken as the last unwelcome option employed only when the last used
location either does not exist at all or exists, but it is not
possible to open the last used mapset or PERMANENT mapset.
 
\vspace{0.3cm}
\begin{figure}[hbt!] 
\begin{center}
\includegraphics[width=17cm]{../pictures/normal_user_diagram2.png} 
\caption[Flowchart of GIS startup for existing users: Proposal 2]{Flowchart of GIS startup for existing users: Proposal 2 (Source: Personal collection)}
\label{fig:normal_user_diagram2}
\end{center}
\end{figure}

\noindent However, the disadvantage of the second solution is that the
GRASS GIS startup process would not be consistent. The first solution
is always consistent - boot in the demolocation in each non-standard
situation and provide the same message in the Info Bar. Since this
massage is closely related to the Data Catalog and data hierarchy in
GRASS GIS in general, we can use the same object of Info Bar which is
used for messages intended for first-time users.

We could also think of other situations where the Info Bar could be
displayed to users (see Figure \ref{fig:other_situations}). For
example, when creating a new location, it can be assumed that the user
will subsequently request the creation of a mapset. Similarly, when
creating a database, it can be assumed that the user will want to
create a new location in the next step. The notifications could
therefore be a kind of guide intended not only for first-time
users. Their purpose would be to speed up the user's work related to
the organization of their data. It could also be the same Info Bar
object placed in the data tab, however, this time the type would not
be informative, but questional.

\vspace{0.3cm}
\begin{figure}[hbt!] 
\begin{center}
\includegraphics[width=17cm]{../pictures/other_situations.png} 
\caption[Flowchart of other situations where the Info Bar could be helpful]{Flowchart of other situations where the Info Bar could be helpful (Source: Personal collection)}
\label{fig:other_situations}
\end{center}
\end{figure}

%% -------<<< Chapter 6: Analysis of the results of the second survey>>>-------\\%%%%%%%%%%%%%%%%%%%%%%%%%%%%%%%%%%%%

\newpage
\vspace*{-1cm}
\fancyhead[RE, RO]{\fancyplain{}{\small \sl{Analysis of the results of the second survey}}}
\section{Analysis of the results of the second survey}
\label{sec:qstat2}

\noindent The second questionnaire called \textbf{Help improve the
  special mode for first-time users} was released on November 26 and
stopped on November 30, 2020. It introduces a new Info Bar solution to
new and existing users and tests its success. At the same time, it
addresses users to share their ideas on how to modify and adapt the
solution so that the resulting solution that emerges from this work,
largely based on this survey, enriches the first-time user experience
with GRASS as much as possible.

The questionnaire is based on a simple task. A user has vector data of
rivers in the Czech Republic in the shapefile format in the coordinate
system S-JTSK/Krovak East North (EPSG:5514) and they would like to
import this data into GRASS and perform a simple task - extract a
river called Otava and save it in a separate layer. Unfortunately, due
to external epidemiological circumstances, this task could not be
tested among GRASS beginners directly on the software. Thus, a survey
was designed that simulates three key situations that, according to
the diagrams in the previous chapter, a new user is likely to get into
and which, according to the results of the second part of the first
survey, are crucial. In these situations, the respondent is presented
with software mockups of Info Bars which (if they are placed in the
right place at the right time and correctly designed in terms of text)
should lead to the right decision on how to continue in work. In the
following text, the term \textit{default location} is preferred to the
term \textit{demolocation}, which is more of a technical (developer)
nature. Semantically, these concepts do not differ.

\textbf{The first situation} shown in the mockup in Figure
\ref{fig:grass_infobar_1} shows the GRASS GIS immediately after
startup. At this point, a user needs to find their way around in a
completely unknown environment as quickly as possible and import their
data so that he can look at it and possibly perform analyzes. The
software starts in the Data tab to the default location, which in the
Map Display window displays the world map in the WGS 84 system
(EPSG:4326). The task of the default location is to give the user a
certain sense of security by showing a map and at the same time an
example of data organization in the Data Catalog, which is now the
center of GRASS. At the first moment, it is therefore essential for
the user to at least passively understand the principle of the GRASS
data hierarchy, which is well visible in the Data Catalog and is also
briefly explained in the first Info Bar. Terms are also specified in
the text using new planned names in parentheses. In order to properly
figure out the first situation, a user must realize that the data he
wants to import into the software has a different coordinate system
(EPSG:5514) than the one defined for the default location
(EPSG:4326). This means that if a user understands the meaning of the
location, he will create a new location that will have the coordinate
system of the data he wants to import.

\vspace{0.3cm}
\begin{figure}[hbt!] 
\begin{center}
\includegraphics[width=17cm]{../pictures/grass_infobar_1.png} 
\caption[Survey 2: Situation 1]{Survey 2: Situation 1 (Source: Personal collection)}
\label{fig:grass_infobar_1}
\end{center}
\end{figure}

\newpage
\noindent Once a user creates a new location, he finds themselves in
\textbf{the second situation} captured in Figure
\ref{fig:grass_infobar_2} which displays the second Info Bar leading
to data import. This Info Bar also explains the concept of PERMANENT
mapset and shows a variant of creating own mapset, which is not
necessary but useful. It means that we are still partially dealing
with the topic of data hierarchy in GRASS, but at the same time, we
advise the user on how to import their data. After successful import,
a map is displayed automatically in the Map Display.

\vspace{0.3cm}
\begin{figure}[hbt!] 
\begin{center}
\includegraphics[width=17cm]{../pictures/grass_infobar_2.png} 
\caption[Survey 2: Situation 2]{Survey 2: Situation 2 (Source: Personal collection)}
\label{fig:grass_infobar_2}
\end{center}
\end{figure}

\noindent The result of Question 5 of the second part of the first
survey shows that users also have a problem with the meaning of
individual tabs. After importing the data, a user wants to either
analyze data directly or change its display in the Map Display
(e.g. change transparency, strength, line color, etc.). In \textbf{the
  third situation} shown in Figure \ref{fig:grass_infobar_3} it is,
therefore, necessary to introduce a user to the two main tabs called
Modules and Display, which are useful for the mentioned purposes. As a
reader probably noticed, the proposal seeks to provide new users with
basic advice on all of the aspects that appeared in Survey 1 Part 2
Question 5.

\vspace{0.3cm}
\begin{figure}[hbt!] 
\begin{center}
\includegraphics[width=17cm]{../pictures/grass_infobar_3.png} 
\caption[Survey 2: Situation 3]{Survey 2: Situation 3 (Source: Personal collection)}
\label{fig:grass_infobar_3}
\end{center}
\end{figure}

\noindent The survey was visited by 32 respondents, 25 of them
provided relevant responses. The completion rate among relevant
respondents was 96 \%. We can see a weekly graph of the number of
responses for a specific day in Figure
\ref{fig:survey1_part2_insight2}.

\vspace{0.3cm}
\begin{figure}[hbt!] 
\begin{center}
\includegraphics[width=10cm]{../surveys/analyzed_data/survey2_insight2.png} 
\caption[Survey 2: Responses by day]{Survey 2: Responses by day (Source: Basic analyzes provided by the SM)}
\label{fig:survey2_insight2}
\end{center}
\end{figure}

\noindent The questionnaire consists of 10 questions - 6 questions
concern the understanding of individual situations, 2 questions are
complementary and two more questions find out how experienced the
interviewer is - both in terms of GIS in general and in terms of GRASS
GIS.

The responses were divided into two groups with regard to the GRASS
experience of individual respondents. The first group called
\textbf{Occasional GRASS users} includes respondents who use GRASS
less than sometimes (they chose less than 50 points in Q10) while the
second group called \textbf{Frequent GRASS users} includes users who
use GRASS more often than sometimes (they selected more than 50 points
or 50 points in Q10).

According to Q9, the \textbf{Occasional GRASS users} group (see Figure
\ref{fig:survey2_respondents_group1}) consists mainly of beginners and
a minority of advanced or intermediate GIS experts (a total of 10
respondents) while the \textbf{Frequent GRASS users} group (see Figure
\ref{fig:survey2_respondents_group2}) includes mainly GIS
professionals and some intermediate (a total of 15 respondents).

\vspace{0.3cm}
\begin{figure}[hbt!] 
\begin{center}
\includegraphics[width=11.5cm]{../surveys/analyzed_data/survey2_respondents_group1.png} 
\caption[Survey 2: GIS proficiency of \textbf{Occasional GRASS users} group]{Survey 2: GIS proficiency of \textbf{Occasional GRASS users} group (Source: Personal collection)}
\label{fig:survey2_respondents_group1}
\end{center}
\end{figure}

\begin{figure}[hbt!] 
\begin{center}
\includegraphics[width=11.5cm]{../surveys/analyzed_data/survey2_respondents_group2.png} 
\caption[Survey 2: GIS proficiency of \textbf{Frequent GRASS users} group]{Survey 2: GIS proficiency of \textbf{Frequent GRASS users} group (Source: Personal collection)}
\label{fig:survey2_respondents_group2}
\end{center}
\end{figure}

\newpage
\noindent Questions Q2, Q4, Q6, and Q8 has a type of Comment Box,
which means that respondents shared their feelings and ideas in the
form of open-ended responses. These are divided into six categories
throughout Survey 2:

\begin{enumerate}
\item Demolocation/Info Bar confused
\item Info Bar could be ignored
\item Terminology should be changed
\item Info Bar too long / Editing of content in Info Bar
\item Info Bar Helped
\item Other
\end{enumerate}

\noindent The answers in Figures \ref{fig:survey2_question2},
\ref{fig:survey2_question4}, \ref{fig:survey2_question6} and
\ref{fig:survey2_question8} are sorted from the most serious category
to the least serious. Only the last sixth category varying in
importance and consists of responses that do not fall into any of the
previous categories. The following lines analyze in detail the answers
from Q1 - Q8 with regard to the user experience that arises from Q9
and Q10.

\par\noindent\rule{\textwidth}{0.4pt}
\noindent \textbf{Question 1: What will be your next step in first situation?}
\par\noindent\rule{\textwidth}{0.4pt}

\noindent There was only one correct choice for this Multiple Choice
answer, namely ``I will create a new Location ''. The first situation
was solved correctly by 17 out of 25 respondents (see Figures
\ref{fig:survey2_question1_histogram_group1},
\ref{fig:survey2_question1_histogram_group2}). In the
\textbf{Occasional GRASS users} group, 5 respondents solved it
correctly, i.e., half of them. Even though the sample of users is very
small, half of the people is definitely a success, because, without
help, a very new or little experienced user probably wouldn't have
thought to create a new location at all.

\par\noindent\rule{\textwidth}{0.4pt}
\noindent \textbf{Question 2: Is something confusing to you? If so, what specifically?? (Optional)}
\par\noindent\rule{\textwidth}{0.4pt}

\noindent This open-ended optional question was answered by 9
respondents and comments can be seen in Figure
\ref{fig:survey2_question2}. As respondent \#21 points out, it is
somewhat confusing that the default location contains two mapsets -
not only the PERMANENT mapset, which must be included in each
location, but also the user-named mapset. The world map is
intentionally placed in the PERMANENT mapset because this mapset is
intended for storing and displaying basic data. However, analyzes
should always be performed in a mapset other than
PERMANENT. Therefore, at the end of GSoC, it was decided that the
default location will also contain a user-named mapset set as
\textbf{current}, which will motivate users not to use PERMANENT
mapset for their further analyzes, but to sort their data into other
mapsets. However, from the point of view of GRASS, this solution does
not make sense, because in all other cases but default location we are
only allowed to display layers from the current mapset, and when
displaying layers from another mapset we need to switch to it. Due to
these circumstances, it would be probably more correct and at the same
time more clearer for first-time users if the default location
contained only the PERMANENT mapset with the included world map. After
all, advice on the use of other mapsets is already part of the Info
Bar in the second situation. Alternatively, a brief explanation of the
term mapset could still be part of the first Info Bar, as suggested by
respondent \#24.

Response \#11 indicates that a user may not realize that he or she is
in a default location at all. This would need to be emphasized
more. Respondents \#13 and \#17 fear that users will ignore the Info
Bar. So another important step would be to think about how to
highlight the Info Bar as much as possible so that most new users will
really read it. However, as we can see from the responses \#10 and
\#24 as well as from the number of correct answers in Q1, the Info
Bar, even in the form in which it appears in this initial proposal, is
certainly an improvement for new GRASS GIS users.

\vspace{0.3cm}
\begin{figure}[hbt!] 
\begin{center}
\includegraphics[width=14cm]{../surveys/analyzed_data/survey2_question1_histogram_group1.png} 
\caption[\textbf{Occasional GRASS users}: Bar Chart  for Question 1 from Survey 2]{\textbf{Occasional GRASS users}: Bar Chart for Question 1 from Survey 2 (Source: Personal collection)}
\label{fig:survey2_question1_histogram_group1}
\end{center}
\end{figure}

\vspace{0.3cm}
\begin{figure}[hbt!] 
\begin{center}
\includegraphics[width=13.5cm]{../surveys/analyzed_data/survey2_question1_histogram_group2.png} 
\caption[\textbf{Frequent GRASS users}: Bar Chart for Question 1 from Survey 2]{\textbf{Frequent GRASS users}: Bar Chart for Question 1 from Survey 2 (Source: Personal collection)}
\label{fig:survey2_question1_histogram_group2}
\end{center}
\end{figure}

\newpage
\vspace*{-1cm}
\par\noindent\rule{\textwidth}{0.4pt}
\noindent \textbf{Question 3: What will be your next step in second situation?}
\par\noindent\rule{\textwidth}{0.4pt}

\noindent There were two correct choices - either import the data
using the \textit{Import vector data} button or go directly to the
File menu, where a user would find the corresponding function called
\textit{Simplified vector import with reprojection (v.import)}. As we
can notice in Figures \ref{fig:survey2_question3_histogram_group1},
\ref{fig:survey2_question3_histogram_group2}, both groups have been
successful, 18 people out of 25 would choose to import data via the
button in the Info Bar, only 5 people would search in the File menu.

\par\noindent\rule{\textwidth}{0.4pt}
\noindent \textbf{Question 4: Is something confusing to you? If so, what specifically?? (Optional)}
\par\noindent\rule{\textwidth}{0.4pt}

\noindent As the comment \#14 in Figure \ref{fig:survey2_question4}
points out, it is not clear from the Info Bar that it is the import
using the \textit{v.import} or \textit{r.import} modules. It is
important to make sure that the user who clicks on the button in the
Info Bar will be able to run this function afterwards without having
to search for it. Therefore, the buttons could contain bitmap images
in addition to the text. Creating a location already has its image
used for management icons, for data import it would be useful to add
basic functions for data import also to the management icons. However,
the question is whether to implement the buttons in the Info Bar at
all. Perhaps it would be clearer to lead a user directly to the File
menu or to the upper toolbar containing management icons, as
respondent \#10 suggests.

\newpage
\vspace{0.3cm}
\begin{figure}[hbt!] 
\begin{center}
\includegraphics[width=15.5cm]{../surveys/analyzed_data/survey2_question2.png} 
\caption[Survey 2 Question 2: Classification of open-ended responses]{Survey 2 Question 2: Classification of open-ended responses (Source: Personal collection)}
\label{fig:survey2_question2}
\end{center}
\end{figure}

\newpage
\vspace{0.3cm}
\begin{figure}[hbt!] 
\begin{center}
\includegraphics[width=14cm]{../surveys/analyzed_data/survey2_question3_histogram_group1.png}
\caption[\textbf{Occasional GRASS users}: Bar Chart for Question 3 from Survey 2]{\textbf{Occasional GRASS users}: Bar Chart for Question 3 from Survey 2 (Source: Personal collection)}
\label{fig:survey2_question3_histogram_group1}
\end{center}
\end{figure}

\vspace{0.3cm}
\begin{figure}[hbt!]
\begin{center}
\includegraphics[width=14cm]{../surveys/analyzed_data/survey2_question3_histogram_group2.png} 
\caption[\textbf{Frequent GRASS users}: Bar Chart for Question 3 from Survey 2]{\textbf{Frequent GRASS users}: Bar Chart for Question 3 from Survey 2 (Source: Personal collection)}
\label{fig:survey2_question3_histogram_group2}
\end{center}
\end{figure}

\noindent The comment number \#13 encounters the problem that the
first help in the Info Bar is not clearly related to the Data
Catalog. Therefore, the Info Bar in the first situation could also
contain small icons next to the explanation of location and mapset,
which are used to distinguish the data hierarchy elements in the Data
Catalog. Another part of the users encounters problematic terminology
in GRASS. In this case, according to the author's opinion, it is worth
waiting for how helpful will the Info Bar be for first-time
users. Only after a certain time when this mechanism will be
functional can we conclude whether it will be necessary to change the
terminology of the data hierarchy or whether the new startup mechanism
is so straightforward to new users that most of them will get their
way around without problems.

\vspace{0.3cm}
\begin{figure}[hbt!] 
\begin{center}
\includegraphics[width=16.5cm]{../surveys/analyzed_data/survey2_question4.png} 
\caption[Survey 2 Question 4: Classification of open-ended responses]{Survey 2 Question 4: Classification of open-ended responses (Source: Personal collection)}
\label{fig:survey2_question4}
\end{center}
\end{figure}

\newpage
\noindent \textbf{Question 5:  What will be your next step in third situation?}
\par\noindent\rule{\textwidth}{0.4pt}

\noindent Two users who selected the Other option, would double-click
on a vector layer to obtain layer properties. It can be concluded that
these users did not understand the function of the Data Catalog. In
the Data Catalog, it is only possible to display the metadata of the
layer, if we want to change the display properties of this layer, we
have to go to the Display tab. It is also interesting that from
\textbf{Occasional GRASS users} only 1 person would go straight to the
Modules tab, while for {Frequent GRASS users} this option prevails. As
we can notice in Figures \ref{fig:survey2_question5_histogram_group1}
and \ref{fig:survey2_question5_histogram_group2}, the half of
\textbf{Occasional GRASS users} would then look to the documentation,
from \textbf{Frequent GRASS users} only three people would do so.

\par\noindent\rule{\textwidth}{0.4pt}
\noindent \textbf{Question 6: Is something confusing to you? If so, what specifically?? (Optional)}
\par\noindent\rule{\textwidth}{0.4pt}

\noindent The first survey shows that people have problems with the
meaning of tabs. Like respondent \#15, also other users, especially
existing ones, may be confused by the purpose of the Data and Display
tabs.

\vspace{0.3cm}
\begin{figure}[hbt!] 
\begin{center}
\includegraphics[width=17cm]{../surveys/analyzed_data/survey2_question5_histogram_group1.png} 
\caption[\textbf{Occasional GRASS users}: Bar Chart for Question 5 from Survey 2]{\textbf{Occasional GRASS users}: Bar Chart for Question 5 from Survey 2 (Source: Personal collection)}
\label{fig:survey2_question5_histogram_group1}
\end{center}
\end{figure}

\newpage
\begin{figure}[hbt!] 
\begin{center}
\includegraphics[width=17cm]{../surveys/analyzed_data/survey2_question5_histogram_group2.png} 
\caption[\textbf{Frequent GRASS users}: Bar Chart for Question 5 from Survey 2]{\textbf{Frequent GRASS users}: Bar Chart for Question 5 from Survey 2 (Source: Personal collection)}
\label{fig:survey2_question5_histogram_group2}
\end{center}
\end{figure}

\begin{figure}[hbt!] 
\begin{center}
\includegraphics[width=14.5cm]{../surveys/analyzed_data/survey2_question6.png} 
\caption[Survey 2 Question 6: Classification of open-ended responses]{Survey 2 Question 6: Classification of open-ended responses (Source: Personal collection)}
\label{fig:survey2_question6}
\end{center}
\end{figure}

\newpage
\noindent The Display tab was only renamed from original
\textit{Layers} tab within GSoC and moved to the second place in the
order of tabs. The Data tab (previously in the fourth place in the tab
order) was placed in the first place. The Data tab has the character
of a data directory and is only used to organize data in GRASS GIS. A
very distant counterpart called Catalog can be found in ArcGIS. Map
layers and their properties are located in the Display tab. In the
Info Bar in the third situation, it would be helpful to better explain
the difference between the Data and Display tabs. Similarly, it could
help to change the terminology of Modules to \textit{Tools}, as
suggested by respondent \#24 in Figure \ref{fig:survey2_question6}.

\par\noindent\rule{\textwidth}{0.4pt}
\noindent \textbf{Question 7: What do you think about the following statement? The advice in Info Bars given in each situation was straightforward and led me very well to the right answers.}
\par\noindent\rule{\textwidth}{0.4pt}

\noindent The agreement with this statement is significantly higher
for the group \textbf{Frequent GRASS users} as seen in Figures
\ref{fig:survey2_question7_histogram_group2} and
\ref{fig:survey2_question7_boxplot_group2}. It was expected since the
perception of these users is somewhat distorted by the knowledge of
GRASS. For the \textbf{Occasional GRASS users}, however, the average
of 53.5 is also not disappointing. Despite the very small number of
respondents, we can conclude that there are more \textbf{Occasional
  GRASS users} who like the Info Bar solution than dislike it.

It is important to admit that the survey indicates what the respondent
should pay attention to. This is not only since a user has a choice of
multiple answers but also to the topic of the survey itself. It is
therefore clear that in practice the success of the Info Bar can be
different.

\vspace{0.3cm}
\begin{figure}[hbt!] 
\begin{center}
\includegraphics[width=10.5cm]{../surveys/analyzed_data/survey2_question7_histogram_group1.png} 
\caption[\textbf{Occasional GRASS users}: Histogram for Question 7 from Survey 2]{\textbf{Occasional GRASS users}: Histogram for Question 7 from Survey 2 (Source: Personal collection)}
\label{fig:survey2_question7_histogram_group1}
\end{center}
\end{figure}

\newpage
\begin{figure}[hbt!] 
\begin{center}
\includegraphics[width=12cm]{../surveys/analyzed_data/survey2_question7_boxplot_group1.png} 
\caption[\textbf{Occasional GRASS users}: Boxplot for Question 7 from Survey 2]{\textbf{Occasional GRASS users}: Boxplot for Question 7 from Survey 2 (Source: Personal collection)}
\label{fig:survey2_question7_boxplot_group1}
\end{center}
\end{figure}

\vspace{0.3cm}
\begin{figure}[hbt!] 
\begin{center}
\includegraphics[width=11cm]{../surveys/analyzed_data/survey2_question7_histogram_group2.png} 
\caption[\textbf{Frequent GRASS users}: Histogram for Question 7 from Survey 2]{\textbf{Frequent GRASS users}: Histogram for Question 7 from Survey 2 (Source: Personal collection)}
\label{fig:survey2_question7_histogram_group2}
\end{center}
\end{figure}

\par\noindent\rule{\textwidth}{0.4pt}
\noindent \textbf{Question 8: Any ideas you want to share? (e.g. the change of wording in Info Bars, adding more information, or, conversely, the removal of some information) (Optional)}
\par\noindent\rule{\textwidth}{0.4pt}

\noindent In addition to the required changes in terminology and
concerns that the Info Bar will be ignored by users, several
respondents said that the texts in the Info Bar are too long. There is
an effort to shorten them as much as possible, but at the same time,
it must contain everything needed. The text of the Info Bar is
definitely not final and will be changed based on proposals in Pull
Requests and also based on longer-term feedback from GRASS users.

A very valuable comment was given by respondent \#18, who suggests
placing the Info Bar above the Data Catalog (not below as presented in
the proposal). This respondent also perceives the Info Bar as a
warning message rather than an information message. It can be evoked
by an orange color that was chosen deliberately in order to highlight
the Info Bar and at the same time to match the color of the selected
item of data hierarchy emphasized by the white text on the orange
field as well. The Info Bar type (for example, information, warnings,
or a question) is distinguished by a small icon on the left side of
the widget. The Info Bar designed for first-time users has the
character of an information message.

\vspace{0.3cm}
\begin{figure}[hbt!] 
\begin{center}
\includegraphics[width=11cm]{../surveys/analyzed_data/survey2_question7_boxplot_group2.png} 
\caption[\textbf{Frequent GRASS users}: Boxplot for Question 7 from Survey 2]{\textbf{Frequent GRASS users}: Boxplot for Question 7 from Survey 2 (Source: Personal collection)}
\label{fig:survey2_question7_boxplot_group2}
\end{center}
\end{figure}

\noindent In terms of concept, the Info Bar in Data Catalog will be
implemented with texts appearing in situations that have been
identified based on the first survey. The concept, therefore, will be
preserved. However, the second survey revealed some shortcomings which
resulted in several changes:

\begin{itemize}
\item Removal of a user-named mapset from the default location
\item Placing the Info Bar above the Data Catalog
\item Text editing in the Info Bar stemming from the second survey
\end {itemize}

\noindent Alternatively, after successful implementation of this foundation, further improvements can be made:

\begin {itemize}
\item Adding basic functions for data import between Management icons
\item Adding bitmap images to the buttons so that users know 100\% what function they are calling through the Info Bar
\end {itemize}

\newpage
\begin{figure}[hbt!] 
\begin{center}
\includegraphics[width=15cm]{../surveys/analyzed_data/survey2_question8.png} 
\caption[Survey 2 Question 8: Classification of open-ended responses]{Survey 2 Question 8: Classification of open-ended responses (Source: Personal collection)}
\label{fig:survey2_question8}
\end{center}
\end{figure}


%% -------<<< Chapter 8: Implementation>>>-------\\%%%%%%%%%%%%%%%%%%%%%%%%%%%%%%%%%%%%
\newpage
\vspace*{-1cm}
\fancyhead[RE, RO]{\fancyplain{}{\small \sl{Implementation}}}
\section{Implementation}
\noindent
\large

\noindent This chapter presents the implementation of first-time mode
from the technical point of view. As mentioned in subsection
\ref{subsection:grassgis} the most changes in GRASS GIS takes place in
Python which also applies to GUI which uses the wxPython
extension. wxPython is a cross-platform toolkit which means that the
same program can be run on multiple platforms without
modification. Currently, supported platforms are Windows, macOS,
Linux, or other Unix-like systems. The resulting design on each
platform can be a bit different as we can also notice when using GRASS
GIS on different systems. We can import wx to a script file as a
package that wraps the GUI components of the popular wxWidgets
cross-platform C++ library established in 1992 at the Artificial
Intelligence Applications Institute at the University of
Edinburgh. There are several other GUI packages for Python, for
example, the Tkinter package is also very popular.

GRASS GIS has been developing since January 2020 on GitHub, a web
service that supports development using the Git versioning tool. This
principle makes working on code much clearer. It stores the history of
work, ensures stylistic consistency using the flake8 command-line
utility, and also allows the creation of Issues, which can have the
character of errors and various improvements. Then these issues are
usually proposed for changes (so-called Pull Requests), which users
discuss. We can say that GitHub partly works as a social network,
which can support the creativity and enthusiasm of developers. In the
following text presenting the changes that have taken place in terms
of the developer's point of view, the term \textit{demolocation} is
preferred to the term \textit{default location}.

\bigskip
\noindent \Large \textbf{Displaying a world map and deleting a user mapset}

\noindent \large Displaying a world map as a part of the default
location is solved in the script \textit{gui/wxpython/lmgr/frame.py}
directly in the constructor of the GMFrame class, which represents the
Layer Manager (see chapter \ref{sec:beforeGSoC}). The map is displayed
whenever the user is in the default location, this is checked via the
location name \textit{world\_latlong\_wgs84}.

The default location (demolocation) is automatically part of the GRASS
GIS distribution. After the first run, this location is copied
on-the-fly to an automatically created database called
``grassdata'. In this demolocation, a mapset named after the user is
created. This described solution was implemented within GSoC. However,
based on Survey 2, it was decided that the user mapset is
confusing. Therefore, the demolocation now starts in the PERMANENT
mapset, which becomes the current mapset at startup. The change mainly
affects the file \textit{gui/wxpython/startup/utils.py}.

\par\noindent\rule{\textwidth}{0.4pt}
\textbf{Pull Requests:}

Displaying a world map: \url{https://github.com/OSGeo/grass/pull/1070}

Deleting a user mapset: \url{https://github.com/OSGeo/grass/pull/1173}
\par\noindent\rule{\textwidth}{0.4pt}

\bigskip
\noindent \Large \textbf{Special mode for first-time users}

\noindent \large The basis of the new special mode for first-time
users is the \texttt{InfoBar} class for which a new file in the path
\textit{gui/wxpython/gui\_core/infobar.py} has been created. The file
was intentionally created in the \textit{gui\_core} directory, as the
\texttt{InfoBar} class is a general template and is therefore usable
for other future InfoBar instances located elsewhere than in the Data
Catalog. This class inherits from the implementation in the Advanced
Generic Widgets (AGW). This package provides many custom wxPython
controls that are simply an addition to the wxPython widgets set. In
terms of Layout and displaying and hiding messages (and buttons), the
Info Bar has been significantly adapted to GRASS GIS.

Next, the \texttt{DataCatalogInfoManager} class was created in the new
script in the path
\textit{gui/wxpython/datacatalog/infomanager.py}. This class contains
methods that define the individual info messages that will be
displayed. It takes care of both the text and the buttons as well as
functions that are called when the buttons are pressed.

Both \texttt{InfoBar} and \texttt{DataCatalogInfoManager} instances
are created in the \texttt{DataCatalog} class in the
\textit{gui/wxpython/datacatalog/catalog.py} file. This class is a
template for an object that is the content of the Data tab. The
\texttt{DataCatalog} class contains first the \texttt{DataCatalogTree}
object, which is often inaccurately called the Data Catalog, secondly
the toolbar object \texttt{DataCatalogToolbar} in which the Management
icons are located, and thirdly the newly implemented \texttt{InfoBar}
object. In Figure \ref{fig:uml_chart} we can see a UML diagram that
shows the classes described above and the relationships between them.

If the user is in the demolocation after startup, the world map is
displayed and at the same time the Info Bar is visible, which displays
advice and buttons for the first situation. The setting of the Info
Bar for the second and third situation is no longer so
straightforward. It is necessary to use Signals from the pydispatch
library.

In the second situation, a Signal is created in the
\texttt{DataCatalogTree} class and emitted if the user is in a
demolocation and a new location is successfully created. At the same
time, when generally creating a new location, there is always a switch
to the PERMANENT mapset of the newly created location, which thus
becomes the current mapset.

\vspace{0.3cm}
\begin{figure}[hbt!] 
\begin{center}
\includegraphics[width=17cm]{../pictures/uml_chart.png} 
\caption[UML diagram of InfoBar implementation]{UML diagram of InfoBar implementation (Source: Personal collection)}
\label{fig:uml_chart}
\end{center}
\end{figure}

\newpage

\noindent The \texttt{showImportDataInfo} method is connected to the
signal in the parent \texttt{DataCatalog} object, which calls
\texttt{InfoManager} with the settings which is set in case of
successful creation of a new location. The Info Bar is displayed in
the second situation after creating a new location using the
appropriate Management icon as well as after creating a location using
the button in the Info Bar set for the first situation.

Showing the Info Bar in the third situation is also handled via Signal
from the pydispatch library. Signal is created in the
\texttt{ImportDialog} class in the
\textit{gui/wxpython/modules/import\_export} file, which is the
parrent for both the \texttt{GdalImportDialog} class for importing
raster data as well as the \texttt{OgrImportDialog} class for
importing vector data. If the import is successful, this Signal is
emited. In the \texttt{DataCatalog} class, the
\texttt{showImportSuccessfulInfo} method calling the
\texttt{InfoManager} with the particular settings, is connected to
this Signal. The Info Bar is displayed in the third situation after
data import using the appropriate Management icons as well as after
data import using the buttons in the Info Bar set for the second
situation.

The Info Bar does not appear if the user creates a location or imports
data from the File tab. In this case, it is assumed that it is not a
first-time user. The Info Bar is therefore only a part of the Data
Catalog, its functionality is not connected with the \texttt{GMFrame}
class, which represents the Layer Manager.

\par\noindent\rule{\textwidth}{0.4pt}
\textbf{Pull Requests:}

Info Bar in first situation: \url{https://github.com/OSGeo/grass/pull/1078}

Info Bar in second situation: \url{https://github.com/OSGeo/grass/pull/1183}

Info Bar in third situation: \url{https://github.com/OSGeo/grass/pull/1204}
\par\noindent\rule{\textwidth}{0.4pt}

\bigskip
\noindent \Large\textbf{Managament icons for vector and raster data import}

\noindent \large Due to the Info Bar in the second situation, two
Management icons for data import are added. They are implemented in
the file \textit{gui/wxpython/datacatalog/toolbar.py} in the
\texttt{DataCatalogToolbar} class and use images used for the same
purpose in QGIS. Depending on the type of import (vector/ raster), the
function called \textit{Simplified vector import with reprojection
  (v.import)} or \textit{Simplified raster import with reprojection
  (v.raster)} is called. These are the same functions that are linked
to the buttons `` Import vector data '' and `` Import raster data ''
in the Info Bar set for the second situation.

\par\noindent\rule{\textwidth}{0.4pt}
\textbf{Pull Request:}

\url{https://github.com/OSGeo/grass/pull/1205}
\par\noindent\rule{\textwidth}{0.4pt}

%% -------<<< Chapter 9: Results >>>-------\\%%%%%%%%%%%%%%%%%%%%%%%%%%%%%%%%%%%%
\newpage
\vspace*{-1cm}
\fancyhead[RE, RO]{\fancyplain{}{\small \sl{Results}}}
\section{Results}
\noindent
\large

The Info Bar is created in Data Catalog each time the user starts
GRASS GIS and it is only up to the situation what message is
displayed. During the work, the user can then close the Info Bar but
in reality it is only hidden and in another defined situation it may
get visible again.  Displaying the world map as part of the
demolocation immediately after GRASS GIS first startup is a
functionality mentioned in responses in Survey 1 Part 2 Question 3,
and developers have long considered this concept.

\vspace{0.3cm}
\begin{figure}[hbt!] 
\begin{center}
\includegraphics[width=17cm]{../pictures/po_prvnich_trech_PRs.png} 
\caption[UML diagram of InfoBar implementation]{UML diagram of InfoBar implementation (Source: Personal collection)}
\label{fig:po_prvnich_trech_PRs}
\end{center}
\end{figure}

\vspace{0.3cm}
\begin{figure}[hbt!] 
\begin{center}
\includegraphics[width=17cm]{../pictures/druha_infobar_a_nove_management_ikonky.png} 
\caption[UML diagram of InfoBar implementation]{UML diagram of InfoBar implementation (Source: Personal collection)}
\label{fig:druha_infobar_a_nove_management_ikonky}
\end{center}
\end{figure}

\vspace{0.3cm}
\begin{figure}[hbt!] 
\begin{center}
\includegraphics[width=17cm]{../pictures/third_infobar.png} 
\caption[UML diagram of InfoBar implementation]{UML diagram of InfoBar implementation (Source: Personal collection)}
\label{fig:third_infobar}
\end{center}
\end{figure}



%% -------<<< Chapter: Discussion >>>-------\\%%%%%%%%%%%%%%%%%%%%%%%%%%%%%%%%%%%%
\newpage
\vspace*{-1cm}
\fancyhead[RE, RO]{\fancyplain{}{\small \sl{Discussion}}}
\section*{Discussion}
\addcontentsline{toc}{section}{Discussion}
\noindent
\large

%% -------<<< Chapter: Conclusion >>>-------\\%%%%%%%%%%%%%%%%%%%%%%%%%%%%%%%%%%%%
\newpage
\vspace*{-1cm}
\fancyhead[RE, RO]{\fancyplain{}{\small \sl{Conclusion}}}
\section*{Conclusion}
\addcontentsline{toc}{section}{Conclusion}
\noindent
\large


%% -------<<< LITERATURA >>>-------\\%%%%%%%%%%%%%%%%%%%%%%%%%%%%%%%%%%%%%%%
\newpage
\vspace*{-6ex}
\renewcommand{\refname}{References} 
\addcontentsline{toc}{section}{References}
\fancyhead[RE, RO]{\fancyplain{}{\small \sl{References}}}
	\selectlanguage{english}
	\bibliographystyle{plain}
	\bibliography{bibliography}

\noindent
\large

\newpage
\vspace*{-1cm}
\appendix
\section{Help improve GRASS GIS startup mechanism and Data Catalog }
\fancyhead[RE, RO]{\fancyplain{}{\small \sl{Appendix A: Questionnaire 1}}}
\label{appendix:A}
\setcounter{page}{1}  % nastaví čítač stránek znovu od jedné
\pagenumbering{Roman} % číslování římskými číslicemi
 
 \begin{figure}[hbt!]
 \begin{center}
 \includegraphics[width=12.5cm]{../surveys/questionnaires/survey1_part1_page1_intro.pdf}
 \end{center}
 \end{figure}
 
 \newpage
 \begin{figure}[hbt!]
 \begin{center}
 \includegraphics[width=15.5cm]{../surveys/questionnaires/survey1_part1_page2_intro.pdf}
 \end{center}
 \end{figure}
  
 \newpage
 \begin{figure}[hbt!]
 \begin{center}
 \includegraphics[width=15.5cm]{../surveys/questionnaires/survey1_part1_page3_questions1_2.pdf}
 \end{center}
 \end{figure}
 
 \newpage
 \begin{figure}[hbt!]
 \begin{center}
 \includegraphics[width=15.5cm]{../surveys/questionnaires/survey1_part1_page4_new_stuff.pdf}
 \end{center}
 \end{figure}
 
 \newpage
 \begin{figure}[hbt!]
 \begin{center}
 \includegraphics[width=15.5cm]{../surveys/questionnaires/survey1_part1_page5_questions3_4.pdf}
 \end{center}
 \end{figure}
 
 \newpage
 \begin{figure}[hbt!]
 \begin{center}
 \includegraphics[width=15.5cm]{../surveys/questionnaires/survey1_part1_page6_questions5_6.pdf}
 \end{center}
 \end{figure}

\newpage
\vspace*{-1cm}
\section{Help create a better first-time user experience in GRASS GIS }
\label{appendix:B}

 \begin{figure}[hbt!]
 \begin{center}
 \includegraphics[width=12cm]{../surveys/questionnaires/survey1_part2_page1_intro.pdf}
 \end{center}
 \end{figure}

 \newpage
 \begin{figure}[hbt!]
 \begin{center}
 \includegraphics[width=15.5cm]{../surveys/questionnaires/survey1_part2_page2_state_after_gsoc.pdf}
 \end{center}
 \end{figure}
 
 \newpage
 \begin{figure}[hbt!]
 \begin{center}
 \includegraphics[width=15.5cm]{../surveys/questionnaires/survey1_part2_page3_questions1.pdf}
 \end{center}
 \end{figure}
 
 \newpage
 \begin{figure}[hbt!]
 \begin{center}
 \includegraphics[width=15.5cm]{../surveys/questionnaires/survey1_part2_page4_questions2_3_4_5.pdf}
 \end{center}
 \end{figure}
  
 \newpage
 \vspace*{-1cm}
 \section{Help improve the special mode for first-time users in GRASS GIS }
 \fancyhead[RE, RO]{\fancyplain{}{\small \sl{Appendix C: Questionnaire 3}}}
 \label{appendix:C}

 \begin{figure}[hbt!]
 \begin{center}
 \includegraphics[width=12cm]{../surveys/questionnaires/survey2-page1_intro.pdf}
 \end{center}
 \end{figure}
 
 \newpage
 \begin{figure}[hbt!]
 \begin{center}
 \includegraphics[width=15.5cm]{../surveys/questionnaires/survey2-page2_task.pdf}
 \end{center}
 \end{figure}
 
 \newpage
 \begin{figure}[hbt!]
 \begin{center}
 \includegraphics[width=15.5cm]{../surveys/questionnaires/survey2-page3_questions1_2.pdf}
 \end{center}
 \end{figure}
 
 \newpage
 \begin{figure}[hbt!]
 \begin{center}
 \includegraphics[width=15.5cm]{../surveys/questionnaires/survey2-page4_questions3_4.pdf}
 \end{center}
 \end{figure}
 
 \newpage
 \begin{figure}[hbt!]
 \begin{center}
 \includegraphics[width=15.5cm]{../surveys/questionnaires/survey2-page5_questions5_6.pdf}
 \end{center}
 \end{figure}
 
 \newpage
 \begin{figure}[hbt!]
 \begin{center}
 \includegraphics[width=15.5cm]{../surveys/questionnaires/survey2-page6_questions7_8.pdf}
 \end{center}
 \end{figure}
 
 \newpage
 \begin{figure}[hbt!]
 \begin{center}
 \includegraphics[width=15.5cm]{../surveys/questionnaires/survey2-page7_final.pdf}
 \end{center}
 \end{figure}

\end{document}
